\documentclass{article}
\usepackage{amsmath}
\usepackage{amsfonts}
\usepackage{amssymb}
\usepackage{geometry}
\usepackage{xcolor}
\geometry{margin=1in}

\title{Rail Geometry Twist Change Calculations - Equivalence Verification}
\author{Rail Geometry Analysis}
\date{}

\begin{document}
\maketitle

\section{Problem Statement}
Given interpolated rail coordinates for left and right rails, verify that two computational approaches for calculating twist parameter changes produce identical results:

\textbf{Parameter-First Approach}: Calculate twist parameters, then compute parameter changes\\
\textbf{Delta-First Approach}: Calculate coordinate deltas, then compute parameter changes directly

This verification demonstrates mathematical equivalence between both computational methods for twist calculations.

\section{Notation}
\textbf{Variable Definitions:}
\begin{itemize}
\item $ch_w$ = Chainage location where twist is calculated
\item $\Delta ch$ = Distance interval over which twist is measured
\item $\tau_0$ = Baseline time (reference measurement)
\item $\tau_m$ = Current measurement time (m = 1, 2, 3, ...)
\item $\text{L}$ = Left rail (subscript)
\item $\text{R}$ = Right rail (subscript)
\item $X, Y, Z$ = 3D coordinates (X = Easting, Y = Northing, Z = Elevation)
\item $\Delta$ = Change or difference operator
\end{itemize}

\textbf{Coordinate Notation:}
\begin{itemize}
\item $Z(ch, \tau_m)_L$ = Z-coordinate of left rail at chainage $ch$ and time $\tau_m$
\item $\Delta Z(ch, \tau_m)_L$ = Change in Z-coordinate: $Z(ch, \tau_m)_L - Z(ch, \tau_0)_L$
\end{itemize}

\textbf{Parameter Notation:}
\begin{itemize}
\item $\text{Cant}(ch, \tau_m)$ = Cant (cross-level) at chainage $ch$ and time $\tau_m$
\item $\text{Twist}(ch_w, \Delta ch, \tau_m)$ = Twist over interval $\Delta ch$ starting at chainage $ch_w$ and time $\tau_m$
\item $\Delta\text{Twist}(ch_w, \Delta ch, \tau_m)$ = Change in twist from baseline to time $\tau_m$
\end{itemize}

\section{Input Dataset}
Assume we have interpolated rail coordinates available at two chainage locations separated by distance $\Delta ch$:
\begin{align}
\text{At } ch_w\text{:} \quad &Z(ch_w, \tau_0)_L, Z(ch_w, \tau_0)_R \quad \text{(baseline)} \\
&Z(ch_w, \tau_m)_L, Z(ch_w, \tau_m)_R \quad \text{(current)} \\[0.5em]
\text{At } ch_w + \Delta ch\text{:} \quad &Z(ch_w + \Delta ch, \tau_0)_L, Z(ch_w + \Delta ch, \tau_0)_R \quad \text{(baseline)} \\
&Z(ch_w + \Delta ch, \tau_m)_L, Z(ch_w + \Delta ch, \tau_m)_R \quad \text{(current)}
\end{align}

\textbf{Example:} At chainages 1000m and 1014m (14m interval), we have baseline and current Z-coordinates for both rails.

\section{Parameter-First Approach}

\subsection{Step 1: Calculate Baseline Cant Values}
\begin{align}
\text{Cant}(ch_w, \tau_0) &= Z(ch_w, \tau_0)_L - Z(ch_w, \tau_0)_R \\
\text{Cant}(ch_w + \Delta ch, \tau_0) &= Z(ch_w + \Delta ch, \tau_0)_L - Z(ch_w + \Delta ch, \tau_0)_R
\end{align}

\subsection{Step 2: Calculate Current Cant Values}
\begin{align}
\text{Cant}(ch_w, \tau_m) &= Z(ch_w, \tau_m)_L - Z(ch_w, \tau_m)_R \\
\text{Cant}(ch_w + \Delta ch, \tau_m) &= Z(ch_w + \Delta ch, \tau_m)_L - Z(ch_w + \Delta ch, \tau_m)_R
\end{align}

\subsection{Step 3: Calculate Baseline and Current Twist}
\begin{align}
\text{Twist}(ch_w, \Delta ch, \tau_0) &= \text{Cant}(ch_w + \Delta ch, \tau_0) - \text{Cant}(ch_w, \tau_0) \\
\text{Twist}(ch_w, \Delta ch, \tau_m) &= \text{Cant}(ch_w + \Delta ch, \tau_m) - \text{Cant}(ch_w, \tau_m)
\end{align}

\subsection{Step 4: Calculate Twist Change}
\begin{equation}
\Delta\text{Twist}(ch_w, \Delta ch, \tau_m)_{\text{param}} = \text{Twist}(ch_w, \Delta ch, \tau_m) - \text{Twist}(ch_w, \Delta ch, \tau_0)
\end{equation}

Substituting the twist definitions:
\begin{align}
\Delta\text{Twist}(ch_w, \Delta ch, \tau_m)_{\text{param}} &= [\text{Cant}(ch_w + \Delta ch, \tau_m) - \text{Cant}(ch_w, \tau_m)] \nonumber \\
&\quad - [\text{Cant}(ch_w + \Delta ch, \tau_0) - \text{Cant}(ch_w, \tau_0)]
\end{align}

\section{Delta-First Approach}

\subsection{Step 1: Calculate Z-Coordinate Deltas}
\begin{align}
\Delta Z(ch_w, \tau_m)_L &= Z(ch_w, \tau_m)_L - Z(ch_w, \tau_0)_L \\
\Delta Z(ch_w, \tau_m)_R &= Z(ch_w, \tau_m)_R - Z(ch_w, \tau_0)_R \\
\Delta Z(ch_w + \Delta ch, \tau_m)_L &= Z(ch_w + \Delta ch, \tau_m)_L - Z(ch_w + \Delta ch, \tau_0)_L \\
\Delta Z(ch_w + \Delta ch, \tau_m)_R &= Z(ch_w + \Delta ch, \tau_m)_R - Z(ch_w + \Delta ch, \tau_0)_R
\end{align}

\subsection{Step 2: Calculate Cant Changes}
\begin{align}
\Delta\text{Cant}(ch_w, \tau_m) &= \Delta Z(ch_w, \tau_m)_L - \Delta Z(ch_w, \tau_m)_R \\
\Delta\text{Cant}(ch_w + \Delta ch, \tau_m) &= \Delta Z(ch_w + \Delta ch, \tau_m)_L - \Delta Z(ch_w + \Delta ch, \tau_m)_R
\end{align}

\subsection{Step 3: Calculate Twist Change Directly}
\begin{equation}
\Delta\text{Twist}(ch_w, \Delta ch, \tau_m)_{\text{delta}} = \Delta\text{Cant}(ch_w + \Delta ch, \tau_m) - \Delta\text{Cant}(ch_w, \tau_m)
\end{equation}

Substituting the cant change definitions:
\begin{align}
\Delta\text{Twist}(ch_w, \Delta ch, \tau_m)_{\text{delta}} &= [\Delta Z(ch_w + \Delta ch, \tau_m)_L - \Delta Z(ch_w + \Delta ch, \tau_m)_R] \nonumber \\
&\quad - [\Delta Z(ch_w, \tau_m)_L - \Delta Z(ch_w, \tau_m)_R]
\end{align}

\section{Equivalence Proof}

\subsection{Expand Parameter-First Method}
Substitute cant definitions into the parameter-first result:
\begin{align}
\Delta\text{Twist}(ch_w, \Delta ch, \tau_m)_{\text{param}} &= [\text{Cant}(ch_w + \Delta ch, \tau_m) - \text{Cant}(ch_w, \tau_m)] \nonumber \\
&\quad - [\text{Cant}(ch_w + \Delta ch, \tau_0) - \text{Cant}(ch_w, \tau_0)]
\end{align}

Expand cant terms:
\begin{align}
&= [[Z(ch_w + \Delta ch, \tau_m)_L - Z(ch_w + \Delta ch, \tau_m)_R] - [Z(ch_w, \tau_m)_L - Z(ch_w, \tau_m)_R]] \nonumber \\
&\quad - [[Z(ch_w + \Delta ch, \tau_0)_L - Z(ch_w + \Delta ch, \tau_0)_R] - [Z(ch_w, \tau_0)_L - Z(ch_w, \tau_0)_R]]
\end{align}

Simplify:
\begin{align}
&= [Z(ch_w + \Delta ch, \tau_m)_L - Z(ch_w + \Delta ch, \tau_m)_R - Z(ch_w, \tau_m)_L + Z(ch_w, \tau_m)_R] \nonumber \\
&\quad - [Z(ch_w + \Delta ch, \tau_0)_L - Z(ch_w + \Delta ch, \tau_0)_R - Z(ch_w, \tau_0)_L + Z(ch_w, \tau_0)_R]
\end{align}

\subsection{Expand Delta-First Method}
Substitute coordinate delta definitions:
\begin{align}
\Delta\text{Twist}(ch_w, \Delta ch, \tau_m)_{\text{delta}} &= [\Delta Z(ch_w + \Delta ch, \tau_m)_L - \Delta Z(ch_w + \Delta ch, \tau_m)_R] \nonumber \\
&\quad - [\Delta Z(ch_w, \tau_m)_L - \Delta Z(ch_w, \tau_m)_R]
\end{align}

Expand delta terms:
\begin{align}
&= [[Z(ch_w + \Delta ch, \tau_m)_L - Z(ch_w + \Delta ch, \tau_0)_L] - [Z(ch_w + \Delta ch, \tau_m)_R - Z(ch_w + \Delta ch, \tau_0)_R]] \nonumber \\
&\quad - [[Z(ch_w, \tau_m)_L - Z(ch_w, \tau_0)_L] - [Z(ch_w, \tau_m)_R - Z(ch_w, \tau_0)_R]]
\end{align}

Simplify:
\begin{align}
&= [Z(ch_w + \Delta ch, \tau_m)_L - Z(ch_w + \Delta ch, \tau_0)_L - Z(ch_w + \Delta ch, \tau_m)_R + Z(ch_w + \Delta ch, \tau_0)_R] \nonumber \\
&\quad - [Z(ch_w, \tau_m)_L - Z(ch_w, \tau_0)_L - Z(ch_w, \tau_m)_R + Z(ch_w, \tau_0)_R]
\end{align}

\subsection{Rearrange Terms}
Rearrange parameter-first result:
\begin{align}
\Delta\text{Twist}(ch_w, \Delta ch, \tau_m)_{\text{param}} &= Z(ch_w + \Delta ch, \tau_m)_L - Z(ch_w + \Delta ch, \tau_0)_L \nonumber \\
&\quad - Z(ch_w + \Delta ch, \tau_m)_R + Z(ch_w + \Delta ch, \tau_0)_R \nonumber \\
&\quad - Z(ch_w, \tau_m)_L + Z(ch_w, \tau_0)_L \nonumber \\
&\quad + Z(ch_w, \tau_m)_R - Z(ch_w, \tau_0)_R
\end{align}

Rearrange delta-first result:
\begin{align}
\Delta\text{Twist}(ch_w, \Delta ch, \tau_m)_{\text{delta}} &= Z(ch_w + \Delta ch, \tau_m)_L - Z(ch_w + \Delta ch, \tau_0)_L \nonumber \\
&\quad - Z(ch_w + \Delta ch, \tau_m)_R + Z(ch_w + \Delta ch, \tau_0)_R \nonumber \\
&\quad - Z(ch_w, \tau_m)_L + Z(ch_w, \tau_0)_L \nonumber \\
&\quad + Z(ch_w, \tau_m)_R - Z(ch_w, \tau_0)_R
\end{align}

\subsection{Final Comparison}
Both expressions are identical:
\begin{equation}
\Delta\text{Twist}(ch_w, \Delta ch, \tau_m)_{\text{param}} = \Delta\text{Twist}(ch_w, \Delta ch, \tau_m)_{\text{delta}}
\end{equation}

\section{Worked Example}
Consider a specific numerical example to demonstrate both computational methods.

\subsection{Given Data}
At chainages $ch_w = 1000$m and $ch_w + \Delta ch = 1014$m ($\Delta ch = 14$m):
\begin{align}
\text{Baseline (}\tau_0\text{):} \quad &Z(1000, \tau_0)_L = 102.345\text{m}, \quad Z(1000, \tau_0)_R = 102.330\text{m} \nonumber \\
&Z(1014, \tau_0)_L = 102.355\text{m}, \quad Z(1014, \tau_0)_R = 102.338\text{m} \\[0.5em]
\text{Current (}\tau_m\text{):} \quad &Z(1000, \tau_m)_L = 102.358\text{m}, \quad Z(1000, \tau_m)_R = 102.340\text{m} \nonumber \\
&Z(1014, \tau_m)_L = 102.368\text{m}, \quad Z(1014, \tau_m)_R = 102.346\text{m}
\end{align}

\subsection{Parameter-First Calculation}
\textbf{Step 1:} Calculate baseline cant values
\begin{align}
\text{Cant}(1000, \tau_0) &= 102.345 - 102.330 = 0.015\text{m} \\
\text{Cant}(1014, \tau_0) &= 102.355 - 102.338 = 0.017\text{m}
\end{align}

\textbf{Step 2:} Calculate current cant values
\begin{align}
\text{Cant}(1000, \tau_m) &= 102.358 - 102.340 = 0.018\text{m} \\
\text{Cant}(1014, \tau_m) &= 102.368 - 102.346 = 0.022\text{m}
\end{align}

\textbf{Step 3:} Calculate baseline and current twist
\begin{align}
\text{Twist}(1000, 14, \tau_0) &= 0.017 - 0.015 = 0.002\text{m} \\
\text{Twist}(1000, 14, \tau_m) &= 0.022 - 0.018 = 0.004\text{m}
\end{align}

\textbf{Step 4:} Calculate twist change
\begin{equation}
\Delta\text{Twist}(1000, 14, \tau_m)_{\text{param}} = 0.004 - 0.002 = 0.002\text{m}
\end{equation}

\subsection{Delta-First Calculation}
\textbf{Step 1:} Calculate coordinate deltas
\begin{align}
\Delta Z(1000, \tau_m)_L &= 102.358 - 102.345 = 0.013\text{m} \\
\Delta Z(1000, \tau_m)_R &= 102.340 - 102.330 = 0.010\text{m} \\
\Delta Z(1014, \tau_m)_L &= 102.368 - 102.355 = 0.013\text{m} \\
\Delta Z(1014, \tau_m)_R &= 102.346 - 102.338 = 0.008\text{m}
\end{align}

\textbf{Step 2:} Calculate cant changes
\begin{align}
\Delta\text{Cant}(1000, \tau_m) &= 0.013 - 0.010 = 0.003\text{m} \\
\Delta\text{Cant}(1014, \tau_m) &= 0.013 - 0.008 = 0.005\text{m}
\end{align}

\textbf{Step 3:} Calculate twist change directly
\begin{equation}
\Delta\text{Twist}(1000, 14, \tau_m)_{\text{delta}} = 0.005 - 0.003 = 0.002\text{m}
\end{equation}

\subsection{Verification}
Both methods yield identical results:
\begin{equation}
\Delta\text{Twist}(1000, 14, \tau_m)_{\text{param}} = \Delta\text{Twist}(1000, 14, \tau_m)_{\text{delta}} = 0.002\text{m}
\end{equation}

\section{Conclusion}
\begin{equation}
\boxed{\Delta\text{Twist}(ch_w, \Delta ch, \tau_m)_{\text{param}} = \Delta\text{Twist}(ch_w, \Delta ch, \tau_m)_{\text{delta}}}
\end{equation}

\textcolor{green}{\textbf{VERIFIED:}} Both approaches produce identical results for twist change calculations.

\textbf{Mathematical Basis:} The equivalence holds because twist is a linear combination of cant values, and the distributive property of subtraction ensures identical results.

\section{Implementation Notes}
\begin{itemize}
\item \textbf{Parameter-First}: More intuitive, provides intermediate cant and twist values for analysis
\item \textbf{Delta-First}: More computationally efficient, reuses coordinate deltas across multiple parameters
\item Both methods are mathematically equivalent and produce identical numerical results
\item Twist calculations require coordinates at two chainage locations separated by the measurement interval
\item Choice depends on computational efficiency needs and whether intermediate parameter values are required
\end{itemize}

\end{document}nt cant values
\begin{align}
\text{Cant}(1000, \tau_m) &= 102.358 - 102.340 = 0.018\text{m} \\
\text{Cant}(1014, \tau_m) &= 102.368 - 102.346 = 0.022\text{m}
\end{align}

\textbf{Step 3:} Calculate baseline and current twist
\begin{align}
\text{Twist}(1000, 14, \tau_0) &= 0.017 - 0.015 = 0.002\text{m} \\
\text{Twist}(1000, 14, \tau_m) &= 0.022 - 0.018 = 0.004\text{m}
\end{align}

\textbf{Step 4:} Calculate twist change
\begin{equation}
\Delta\text{Twist}(1000, 14, \tau_m)_{\text{param}} = 0.004 - 0.002 = 0.002\text{m}
\end{equation}

\subsection{Delta-First Calculation}
\textbf{Step 1:} Calculate coordinate deltas
\begin{align}
\Delta Z(1000, \tau_m)_L &= 102.358 - 102.345 = 0.013\text{m} \\
\Delta Z(1000, \tau_m)_R &= 102.340 - 102.330 = 0.010\text{m} \\
\Delta Z(1014, \tau_m)_L &= 102.368 - 102.355 = 0.013\text{m} \\
\Delta Z(1014, \tau_m)_R &= 102.346 - 102.338 = 0.008\text{m}
\end{align}

\textbf{Step 2:} Calculate cant changes
\begin{align}
\Delta\text{Cant}(1000, \tau_m) &= 0.013 - 0.010 = 0.003\text{m} \\
\Delta\text{Cant}(1014, \tau_m) &= 0.013 - 0.008 = 0.005\text{m}
\end{align}

\textbf{Step 3:} Calculate twist change directly
\begin{equation}
\Delta\text{Twist}(1000, 14, \tau_m)_{\text{delta}} = 0.005 - 0.003 = 0.002\text{m}
\end{equation}

\subsection{Verification}
Both methods yield identical results:
\begin{equation}
\Delta\text{Twist}(1000, 14, \tau_m)_{\text{param}} = \Delta\text{Twist}(1000, 14, \tau_m)_{\text{delta}} = 0.002\text{m}
\end{equation}nt cant values
\begin{align}
\text{Cant}(1000, \tau_m) &= 102.358 - 102.340 = 0.018\text{m} \\
\text{Cant}(1020, \tau_m) &= 102.365 - 102.342 = 0.023\text{m}
\end{align}

\textbf{Step 3:} Calculate baseline and current twist
\begin{align}
\text{Twist}(1000, 20, \tau_0) &= 0.020 - 0.015 = 0.005\text{m} \\
\text{Twist}(1000, 20, \tau_m) &= 0.023 - 0.018 = 0.005\text{m}
\end{align}

\textbf{Step 4:} Calculate twist change
\begin{equation}
\Delta\text{Twist}(1000, 20, \tau_m)_{\text{param}} = 0.005 - 0.005 = 0.000\text{m}
\end{equation}

\subsection{Delta-First Calculation}
\textbf{Step 1:} Calculate coordinate deltas
\begin{align}
\Delta Z(1000, \tau_m)_L &= 102.358 - 102.345 = 0.013\text{m} \\
\Delta Z(1000, \tau_m)_R &= 102.340 - 102.330 = 0.010\text{m} \\
\Delta Z(1020, \tau_m)_L &= 102.365 - 102.355 = 0.010\text{m} \\
\Delta Z(1020, \tau_m)_R &= 102.342 - 102.335 = 0.007\text{m}
\end{align}

\textbf{Step 2:} Calculate cant changes
\begin{align}
\Delta\text{Cant}(1000, \tau_m) &= 0.013 - 0.010 = 0.003\text{m} \\
\Delta\text{Cant}(1020, \tau_m) &= 0.010 - 0.007 = 0.003\text{m}
\end{align}

\textbf{Step 3:} Calculate twist change directly
\begin{equation}
\Delta\text{Twist}(1000, 20, \tau_m)_{\text{delta}} = 0.003 - 0.003 = 0.000\text{m}
\end{equation}

\subsection{Verification}
Both methods yield identical results:
\begin{equation}
\Delta\text{Twist}(1000, 20, \tau_m)_{\text{param}} = \Delta\text{Twist}(1000, 20, \tau_m)_{\text{delta}} = 0.000\text{m}
\end{equation}

\section{Conclusion}
\begin{equation}
\boxed{\Delta\text{Twist}(ch_w, \Delta ch, \tau_m)_{\text{param}} = \Delta\text{Twist}(ch_w, \Delta ch, \tau_m)_{\text{delta}}}
\end{equation}

\textcolor{green}{\textbf{VERIFIED:}} Both approaches produce identical results for twist change calculations.

\textbf{Mathematical Basis:} The equivalence holds because twist is a linear combination of cant values, and the distributive property of subtraction ensures identical results.

\section{Implementation Notes}
\begin{itemize}
\item \textbf{Parameter-First}: More intuitive, provides intermediate cant and twist values for analysis
\item \textbf{Delta-First}: More computationally efficient, reuses coordinate deltas across multiple parameters
\item Both methods are mathematically equivalent and produce identical numerical results
\item Twist calculations require coordinates at two chainage locations separated by the measurement interval
\item Choice depends on computational efficiency needs and whether intermediate parameter values are required
\end{itemize}

\end{document}