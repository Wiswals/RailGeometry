\documentclass{article}
\usepackage{amsmath}
\usepackage{amsfonts}
\usepackage{amssymb}
\usepackage{geometry}
\usepackage{xcolor}
\geometry{margin=1in}

\title{Rail Geometry Cant Change Calculations - Equivalence Verification}
\author{Rail Geometry Analysis}
\date{}

\begin{document}
\maketitle

\section{Problem Statement}
Given interpolated rail coordinates for left and right rails, verify that two computational approaches for calculating cant parameter changes produce identical results:

\textbf{Parameter-First Approach}: Calculate cant parameters, then compute parameter changes\\
\textbf{Delta-First Approach}: Calculate coordinate deltas, then compute parameter changes directly

This verification demonstrates mathematical equivalence between both computational methods for cant calculations.

\section{Notation}
\textbf{Variable Definitions:}
\begin{itemize}
\item $ch_c$ = Chainage location (distance along track centerline) where cant is calculated
\item $\tau_0$ = Baseline time (reference measurement)
\item $\tau_m$ = Current measurement time (m = 1, 2, 3, ...)
\item $\text{L}$ = Left rail (subscript)
\item $\text{R}$ = Right rail (subscript)
\item $X, Y, Z$ = 3D coordinates (X = Easting, Y = Northing, Z = Elevation)
\item $\Delta$ = Change or difference operator
\end{itemize}

\textbf{Coordinate Notation:}
\begin{itemize}
\item $Z(ch_c, \tau_m)_L$ = Z-coordinate of left rail at chainage $ch_c$ and time $\tau_m$
\item $\Delta Z(ch_c, \tau_m)_L$ = Change in Z-coordinate: $Z(ch_c, \tau_m)_L - Z(ch_c, \tau_0)_L$
\end{itemize}

\textbf{Parameter Notation:}
\begin{itemize}
\item $\text{Cant}(ch_c, \tau_m)$ = Cant (cross-level) at chainage $ch_c$ and time $\tau_m$
\item $\Delta\text{Cant}(ch_c, \tau_m)$ = Change in cant from baseline to time $\tau_m$
\end{itemize}

\section{Input Dataset}
Assume we have interpolated rail coordinates available at a specific chainage location:
\begin{align}
\text{Left Rail:} \quad &Z(ch_c, \tau_0)_L \quad \text{(baseline measurement)} \\
&Z(ch_c, \tau_m)_L \quad \text{(current measurement)} \\[0.5em]
\text{Right Rail:} \quad &Z(ch_c, \tau_0)_R \quad \text{(baseline measurement)} \\
&Z(ch_c, \tau_m)_R \quad \text{(current measurement)}
\end{align}

\textbf{Example:} At chainage 1000m, we have baseline Z-coordinates from January 2024 ($\tau_0$) and current Z-coordinates from June 2024 ($\tau_m$) for both left and right rails.

\section{Parameter-First Approach}

\subsection{Step 1: Calculate Baseline Cant}
\begin{equation}
\text{Cant}(ch_c, \tau_0) = Z(ch_c, \tau_0)_L - Z(ch_c, \tau_0)_R
\end{equation}

\subsection{Step 2: Calculate Current Cant}
\begin{equation}
\text{Cant}(ch_c, \tau_m) = Z(ch_c, \tau_m)_L - Z(ch_c, \tau_m)_R
\end{equation}

\subsection{Step 3: Calculate Cant Change}
\begin{equation}
\Delta\text{Cant}(ch_c, \tau_m)_{\text{param}} = \text{Cant}(ch_c, \tau_m) - \text{Cant}(ch_c, \tau_0)
\end{equation}

Substituting the cant definitions:
\begin{align}
\Delta\text{Cant}(ch_c, \tau_m)_{\text{param}} &= [Z(ch_c, \tau_m)_L - Z(ch_c, \tau_m)_R] \nonumber \\
&\quad - [Z(ch_c, \tau_0)_L - Z(ch_c, \tau_0)_R]
\end{align}

\section{Delta-First Approach}

\subsection{Step 1: Calculate Z-Coordinate Deltas}
\begin{align}
\Delta Z(ch_c, \tau_m)_L &= Z(ch_c, \tau_m)_L - Z(ch_c, \tau_0)_L \\
\Delta Z(ch_c, \tau_m)_R &= Z(ch_c, \tau_m)_R - Z(ch_c, \tau_0)_R
\end{align}

\subsection{Step 2: Calculate Cant Change Directly}
\begin{equation}
\Delta\text{Cant}(ch_c, \tau_m)_{\text{delta}} = \Delta Z(ch_c, \tau_m)_L - \Delta Z(ch_c, \tau_m)_R
\end{equation}

Substituting the delta definitions:
\begin{align}
\Delta\text{Cant}(ch_c, \tau_m)_{\text{delta}} &= [Z(ch_c, \tau_m)_L - Z(ch_c, \tau_0)_L] \nonumber \\
&\quad - [Z(ch_c, \tau_m)_R - Z(ch_c, \tau_0)_R]
\end{align}

\section{Equivalence Proof}

\subsection{Algebraic Expansion}
Expand parameter-first result:
\begin{align}
\Delta\text{Cant}(ch_c, \tau_m)_{\text{param}} &= [Z(ch_c, \tau_m)_L - Z(ch_c, \tau_m)_R] \\
&\quad - [Z(ch_c, \tau_0)_L - Z(ch_c, \tau_0)_R] \\
&= Z(ch_c, \tau_m)_L - Z(ch_c, \tau_m)_R \\
&\quad - Z(ch_c, \tau_0)_L + Z(ch_c, \tau_0)_R
\end{align}

Expand delta-first result:
\begin{align}
\Delta\text{Cant}(ch_c, \tau_m)_{\text{delta}} &= [Z(ch_c, \tau_m)_L - Z(ch_c, \tau_0)_L] \\
&\quad - [Z(ch_c, \tau_m)_R - Z(ch_c, \tau_0)_R] \\
&= Z(ch_c, \tau_m)_L - Z(ch_c, \tau_0)_L \\
&\quad - Z(ch_c, \tau_m)_R + Z(ch_c, \tau_0)_R
\end{align}

\subsection{Rearranging Terms}
Rearrange parameter-first:
\begin{align}
\Delta\text{Cant}(ch_c, \tau_m)_{\text{param}} &= Z(ch_c, \tau_m)_L - Z(ch_c, \tau_0)_L \nonumber \\
&\quad - Z(ch_c, \tau_m)_R + Z(ch_c, \tau_0)_R
\end{align}

Compare with delta-first:
\begin{align}
\Delta\text{Cant}(ch_c, \tau_m)_{\text{delta}} &= Z(ch_c, \tau_m)_L - Z(ch_c, \tau_0)_L \nonumber \\
&\quad - Z(ch_c, \tau_m)_R + Z(ch_c, \tau_0)_R
\end{align}

\subsection{Final Comparison}
Both expressions are identical:
\begin{equation}
\Delta\text{Cant}(ch_c, \tau_m)_{\text{param}} = \Delta\text{Cant}(ch_c, \tau_m)_{\text{delta}}
\end{equation}

\section{Worked Example}
Consider a specific numerical example to demonstrate both computational methods.

\subsection{Given Data}
At chainage $ch_c = 1000$m:
\begin{align}
\text{Baseline (}\tau_0\text{):} \quad &Z(1000, \tau_0)_L = 102.345\text{m} \nonumber \\
&Z(1000, \tau_0)_R = 102.330\text{m} \\[0.5em]
\text{Current (}\tau_m\text{):} \quad &Z(1000, \tau_m)_L = 102.358\text{m} \nonumber \\
&Z(1000, \tau_m)_R = 102.340\text{m}
\end{align}

\subsection{Parameter-First Calculation}
\textbf{Step 1:} Calculate baseline cant
\begin{equation}
\text{Cant}(1000, \tau_0) = 102.345 - 102.330 = 0.015\text{m}
\end{equation}

\textbf{Step 2:} Calculate current cant
\begin{equation}
\text{Cant}(1000, \tau_m) = 102.358 - 102.340 = 0.018\text{m}
\end{equation}

\textbf{Step 3:} Calculate cant change
\begin{equation}
\Delta\text{Cant}(1000, \tau_m)_{\text{param}} = 0.018 - 0.015 = 0.003\text{m}
\end{equation}

\subsection{Delta-First Calculation}
\textbf{Step 1:} Calculate coordinate deltas
\begin{align}
\Delta Z(1000, \tau_m)_L &= 102.358 - 102.345 = 0.013\text{m} \\
\Delta Z(1000, \tau_m)_R &= 102.340 - 102.330 = 0.010\text{m}
\end{align}

\textbf{Step 2:} Calculate cant change directly
\begin{equation}
\Delta\text{Cant}(1000, \tau_m)_{\text{delta}} = 0.013 - 0.010 = 0.003\text{m}
\end{equation}

\subsection{Verification}
Both methods yield identical results:
\begin{equation}
\Delta\text{Cant}(1000, \tau_m)_{\text{param}} = \Delta\text{Cant}(1000, \tau_m)_{\text{delta}} = 0.003\text{m}
\end{equation}

\section{Conclusion}
\begin{equation}
\boxed{\Delta\text{Cant}(ch_c, \tau_m)_{\text{param}} = \Delta\text{Cant}(ch_c, \tau_m)_{\text{delta}}}
\end{equation}

\textcolor{green}{\textbf{VERIFIED:}} Both approaches produce identical results for cant change calculations.

\textbf{Mathematical Basis:} The equivalence holds due to the linear nature of cant calculations and the distributive property of subtraction over addition.

\section{Implementation Notes}
\begin{itemize}
\item \textbf{Parameter-First}: More intuitive, provides intermediate cant values for analysis
\item \textbf{Delta-First}: More computationally efficient, reuses coordinate deltas
\item Both methods are mathematically equivalent and produce identical numerical results
\item Choice depends on computational efficiency needs and whether intermediate cant values are required
\end{itemize}

\end{document}