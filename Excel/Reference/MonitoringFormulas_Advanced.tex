\documentclass{article}
\usepackage{amsmath}
\usepackage{amsfonts}
\usepackage{amssymb}
\usepackage{geometry}
\usepackage{xcolor}
\geometry{margin=1in}

% Custom mathematical operators
\DeclareMathOperator{\Cant}{Cant}
\DeclareMathOperator{\Gauge}{Gauge}
\DeclareMathOperator{\Twist}{Twist}
\DeclareMathOperator{\Versine}{Versine}
\DeclareMathOperator{\Offset}{Offset}

% Custom commands for coordinates
\newcommand{\coord}[3]{(#1, #2, #3)}
\newcommand{\coordxy}[2]{(#1, #2)}
\newcommand{\prism}[1]{P_{#1}}
\newcommand{\time}[1]{\tau_{#1}}
\newcommand{\chainage}[1]{ch_{#1}}

% Custom commands for interpolated coordinates
\newcommand{\Xintp}[2]{X_{\text{intp}}(#1, #2)}
\newcommand{\Yintp}[2]{Y_{\text{intp}}(#1, #2)}
\newcommand{\Zintp}[2]{Z_{\text{intp}}(#1, #2)}

% Custom commands for rail sides
\newcommand{\Left}{\text{L}}
\newcommand{\Right}{\text{R}}

% Custom commands for deltas
\newcommand{\DeltaX}[2]{\Delta X(#1, #2)}
\newcommand{\DeltaY}[2]{\Delta Y(#1, #2)}
\newcommand{\DeltaZ}[2]{\Delta Z(#1, #2)}

% Custom commands for offsets
\newcommand{\Xoff}[1]{\Delta X_{\text{off}}(#1)}
\newcommand{\Yoff}[1]{\Delta Y_{\text{off}}(#1)}
\newcommand{\Zoff}[1]{\Delta Z_{\text{off}}(#1)}

% Distance calculations
\newcommand{\distance}[6]{\sqrt{(#1-#4)^2 + (#2-#5)^2 + (#3-#6)^2}}
\newcommand{\hdistance}[4]{\sqrt{(#1-#3)^2 + (#2-#4)^2}}

% Baseline coordinates
\newcommand{\Xbase}[1]{X_0(#1)}
\newcommand{\Ybase}[1]{Y_0(#1)}
\newcommand{\Zbase}[1]{Z_0(#1)}

\title{Rail Geometry Monitoring - Advanced Mathematical Framework}
\author{Rail Geometry Analysis}
\date{}

\begin{document}
\maketitle

\section{Mathematical Foundation}

\subsection{Coordinate System Definition}
Let us define the fundamental coordinate system for prism $\prism{p}$ at time $\time{m}$:
\begin{equation}
\coord{X(\prism{p}, \time{m})}{Y(\prism{p}, \time{m})}{Z(\prism{p}, \time{m})}
\end{equation}

\subsection{Baseline Establishment}
The baseline coordinates are established at $\time{0}$:
\begin{align}
\Xbase{\prism{p}} &= X(\prism{p}, \time{0}) \\
\Ybase{\prism{p}} &= Y(\prism{p}, \time{0}) \\
\Zbase{\prism{p}} &= Z(\prism{p}, \time{0})
\end{align}

Building upon this foundation, we define displacement vectors:
\begin{align}
\DeltaX{\prism{p}}{\time{m}} &= X(\prism{p}, \time{m}) - \Xbase{\prism{p}} \\
\DeltaY{\prism{p}}{\time{m}} &= Y(\prism{p}, \time{m}) - \Ybase{\prism{p}} \\
\DeltaZ{\prism{p}}{\time{m}} &= Z(\prism{p}, \time{m}) - \Zbase{\prism{p}}
\end{align}

\section{Coordinate Transformation}

\subsection{Prism-to-Rail Transformation}
Transform prism coordinates to rail running edge using offset corrections:
\begin{align}
X_{\text{rail}}(\prism{p}, \time{m}) &= X_{\text{prism}}(\prism{p}, \time{m}) + \Xoff{\prism{p}} \\
Y_{\text{rail}}(\prism{p}, \time{m}) &= Y_{\text{prism}}(\prism{p}, \time{m}) + \Yoff{\prism{p}} \\
Z_{\text{rail}}(\prism{p}, \time{m}) &= Z_{\text{prism}}(\prism{p}, \time{m}) + \Zoff{\prism{p}}
\end{align}

\section{Interpolation Framework}

\subsection{Bounding Point Selection}
For target chainage $\chainage{\text{target}}$, find bounding points:
\begin{align}
\chainage{\text{last}} &= \max\{\chainage{p} : \chainage{p} \leq \chainage{\text{target}}\} \\
\chainage{\text{next}} &= \min\{\chainage{p} : \chainage{p} \geq \chainage{\text{target}}\}
\end{align}

\subsection{Interpolation Ratio}
Define the fundamental interpolation parameter:
\begin{equation}
r = \frac{\chainage{\text{target}} - \chainage{\text{last}}}{\chainage{\text{next}} - \chainage{\text{last}}}
\end{equation}

\subsection{Linear Interpolation}
Using the ratio $r$, interpolated coordinates become:
\begin{align}
\Xintp{\chainage{\text{target}}}{\time{m}} &= X_{\text{last}} + r \cdot (X_{\text{next}} - X_{\text{last}}) \\
\Yintp{\chainage{\text{target}}}{\time{m}} &= Y_{\text{last}} + r \cdot (Y_{\text{next}} - Y_{\text{last}}) \\
\Zintp{\chainage{\text{target}}}{\time{m}} &= Z_{\text{last}} + r \cdot (Z_{\text{next}} - Z_{\text{last}})
\end{align}

\section{Geometry Parameter Calculations}

\subsection{Cant Definition}
For corresponding left and right rail points at chainage $\chainage{c}$:
\begin{equation}
\Cant(\chainage{c}, \time{m}) = \Zintp{\chainage{c}}{\time{m}}_{\Left} - \Zintp{\chainage{c}}{\time{m}}_{\Right}
\end{equation}

\subsection{Gauge Calculations}
\subsubsection{3D Gauge (True Spatial Distance)}
\begin{equation}
\Gauge_{3D}(\chainage{c}, \time{m}) = \distance{\Xintp{\chainage{c}}{\time{m}}_{\Left}}{\Yintp{\chainage{c}}{\time{m}}_{\Left}}{\Zintp{\chainage{c}}{\time{m}}_{\Left}}{\Xintp{\chainage{c}}{\time{m}}_{\Right}}{\Yintp{\chainage{c}}{\time{m}}_{\Right}}{\Zintp{\chainage{c}}{\time{m}}_{\Right}}
\end{equation}

\subsubsection{2D Gauge (Traditional Horizontal)}
\begin{equation}
\Gauge_{2D}(\chainage{c}, \time{m}) = \hdistance{\Xintp{\chainage{c}}{\time{m}}_{\Left}}{\Yintp{\chainage{c}}{\time{m}}_{\Left}}{\Xintp{\chainage{c}}{\time{m}}_{\Right}}{\Yintp{\chainage{c}}{\time{m}}_{\Right}}
\end{equation}

\subsection{Twist Calculation}
Building on the cant definition, twist over interval $\Delta ch$ is:
\begin{equation}
\Twist(\chainage{w}, \Delta ch, \time{m}) = \Cant(\chainage{w} + \Delta ch, \time{m}) - \Cant(\chainage{w}, \time{m})
\end{equation}

The twist rate per unit distance becomes:
\begin{equation}
\frac{d\Cant}{dch}(\chainage{w}, \time{m}) = \frac{\Twist(\chainage{w}, \Delta ch, \time{m})}{\Delta ch}
\end{equation}

\section{Versine Analysis}

\subsection{Horizontal Versine}
For three points at chainages $\chainage{v-s}$, $\chainage{v}$, $\chainage{v+s}$:

\subsubsection{Perpendicular Offset}
\begin{equation}
\Offset_{\text{hz}}(\chainage{v}) = \frac{|(Y_{v+s} - Y_{v-s}) \cdot X_v - (X_{v+s} - X_{v-s}) \cdot Y_v + X_{v+s} \cdot Y_{v-s} - X_{v-s} \cdot Y_{v+s}|}{\hdistance{X_{v+s}}{Y_{v+s}}{X_{v-s}}{Y_{v-s}}}
\end{equation}

\subsubsection{Signed Horizontal Versine}
\begin{equation}
\Versine_{\text{hz}}(\chainage{v}) = \Offset_{\text{hz}}(\chainage{v}) \times \text{sign}
\end{equation}

\subsection{Vertical Versine}
Using chainage-based interpolation for the expected elevation:
\begin{equation}
Z_{\text{expected}} = Z_{v-s} + \frac{\chainage{v} - \chainage{v-s}}{\chainage{v+s} - \chainage{v-s}} \cdot (Z_{v+s} - Z_{v-s})
\end{equation}

Therefore:
\begin{equation}
\Versine_{\text{vt}}(\chainage{v}) = Z_v - Z_{\text{expected}}
\end{equation}

\section{Change Analysis Framework}

\subsection{Parameter Change Definitions}
For any geometry parameter $P$, define baseline and change:
\begin{align}
P_0(\chainage{}) &= P(\chainage{}, \time{0}) \\
\Delta P(\chainage{}, \time{m}) &= P(\chainage{}, \time{m}) - P_0(\chainage{})
\end{align}

\subsection{Specific Parameter Changes}
\begin{align}
\Delta\Cant(\chainage{c}, \time{m}) &= \Cant(\chainage{c}, \time{m}) - \Cant(\chainage{c}, \time{0}) \\
\Delta\Gauge(\chainage{c}, \time{m}) &= \Gauge(\chainage{c}, \time{m}) - \Gauge(\chainage{c}, \time{0}) \\
\Delta\Twist(\chainage{w}, \time{m}) &= \Twist(\chainage{w}, \time{m}) - \Twist(\chainage{w}, \time{0})
\end{align}

\subsection{Rate of Change}
The temporal rate of parameter change:
\begin{equation}
\frac{dP}{dt}(\chainage{}, \time{m}) = \frac{\Delta P(\chainage{}, \time{m})}{\time{m} - \time{0}}
\end{equation}

\subsection{Threshold Analysis}
Define alert conditions:
\begin{equation}
\text{Alert}_P(\chainage{}, \time{m}) = \begin{cases}
\textcolor{red}{\text{CRITICAL}} & \text{if } |\Delta P(\chainage{}, \time{m})| > T_{\text{critical}} \\
\textcolor{orange}{\text{WARNING}} & \text{if } |\Delta P(\chainage{}, \time{m})| > T_{\text{warning}} \\
\textcolor{green}{\text{NORMAL}} & \text{otherwise}
\end{cases}
\end{equation}

\section{Summary}
This advanced mathematical framework provides a systematic approach to rail geometry monitoring using:
\begin{itemize}
\item Custom LaTeX commands for consistent notation
\item Progressive formula building from basic definitions
\item Reusable mathematical operators
\item Comprehensive change analysis framework
\end{itemize}

The framework enables automated calculation of all geometry parameters and their changes over time for effective rail infrastructure monitoring.

\end{document}