\documentclass{article}
\usepackage{amsmath}
\usepackage{amsfonts}
\usepackage{amssymb}
\usepackage{geometry}
\usepackage{xcolor}
\geometry{margin=1in}

% Custom commands for simplified notation
\newcommand{\chainage}[1]{ch_{#1}}
\newcommand{\time}[1]{\tau_{#1}}
\newcommand{\Left}{\text{L}}
\newcommand{\Right}{\text{R}}

% Coordinate commands
\newcommand{\Z}[3]{Z(#1, #2)_{#3}}

% Delta commands
\newcommand{\DZ}[3]{\Delta Z(#1, #2)_{#3}}

% Cant and Twist commands
\newcommand{\Cant}[2]{\text{Cant}(#1, #2)}
\newcommand{\Twist}[3]{\text{Twist}(#1, #2, #3)}
\newcommand{\DTwist}[3]{\Delta\text{Twist}(#1, #2, #3)}

\title{Twist Change Calculation Methods - Equivalence Verification}
\author{Rail Geometry Analysis}
\date{}

\begin{document}
\maketitle

\section{Problem Statement}
Given interpolated rail coordinates at chainages $\chainage{w}$ and $\chainage{w} + \Delta ch$ for left and right rails, verify that two methods for calculating twist change produce identical results:

\textbf{Method A}: Calculate twist, then compute change in twist\\
\textbf{Method B}: Calculate coordinate deltas, then compute twist change directly

\section{Input Dataset}
Assume we have interpolated rail coordinates at two chainages separated by interval $\Delta ch$:

\textbf{At chainage} $\chainage{w}$:
\begin{align}
\text{Left Rail:} \quad &\Z{\chainage{w}}{\time{0}}{\Left}, \Z{\chainage{w}}{\time{m}}{\Left} \\
\text{Right Rail:} \quad &\Z{\chainage{w}}{\time{0}}{\Right}, \Z{\chainage{w}}{\time{m}}{\Right}
\end{align}

\textbf{At chainage} $\chainage{w} + \Delta ch$:
\begin{align}
\text{Left Rail:} \quad &\Z{\chainage{w} + \Delta ch}{\time{0}}{\Left}, \Z{\chainage{w} + \Delta ch}{\time{m}}{\Left} \\
\text{Right Rail:} \quad &\Z{\chainage{w} + \Delta ch}{\time{0}}{\Right}, \Z{\chainage{w} + \Delta ch}{\time{m}}{\Right}
\end{align}

\section{Method A: Twist-First Approach}

\subsection{Step 1: Calculate Baseline Twist}
First, calculate cant at both chainages for baseline:
\begin{align}
\Cant{\chainage{w}}{\time{0}} &= \Z{\chainage{w}}{\time{0}}{\Left} - \Z{\chainage{w}}{\time{0}}{\Right} \\
\Cant{\chainage{w} + \Delta ch}{\time{0}} &= \Z{\chainage{w} + \Delta ch}{\time{0}}{\Left} - \Z{\chainage{w} + \Delta ch}{\time{0}}{\Right}
\end{align}

Then calculate baseline twist:
\begin{equation}
\Twist{\chainage{w}}{\Delta ch}{\time{0}} = \Cant{\chainage{w} + \Delta ch}{\time{0}} - \Cant{\chainage{w}}{\time{0}}
\end{equation}

\subsection{Step 2: Calculate Current Twist}
Calculate cant at both chainages for current time:
\begin{align}
\Cant{\chainage{w}}{\time{m}} &= \Z{\chainage{w}}{\time{m}}{\Left} - \Z{\chainage{w}}{\time{m}}{\Right} \\
\Cant{\chainage{w} + \Delta ch}{\time{m}} &= \Z{\chainage{w} + \Delta ch}{\time{m}}{\Left} - \Z{\chainage{w} + \Delta ch}{\time{m}}{\Right}
\end{align}

Then calculate current twist:
\begin{equation}
\Twist{\chainage{w}}{\Delta ch}{\time{m}} = \Cant{\chainage{w} + \Delta ch}{\time{m}} - \Cant{\chainage{w}}{\time{m}}
\end{equation}

\subsection{Step 3: Calculate Twist Change}
\begin{equation}
\DTwist{\chainage{w}}{\Delta ch}{\time{m}}_A = \Twist{\chainage{w}}{\Delta ch}{\time{m}} - \Twist{\chainage{w}}{\Delta ch}{\time{0}}
\end{equation}

Substituting the twist definitions:
\begin{align}
\DTwist{\chainage{w}}{\Delta ch}{\time{m}}_A &= \left[\Cant{\chainage{w} + \Delta ch}{\time{m}} - \Cant{\chainage{w}}{\time{m}}\right] \\
&\quad - \left[\Cant{\chainage{w} + \Delta ch}{\time{0}} - \Cant{\chainage{w}}{\time{0}}\right]
\end{align}

\section{Method B: Delta-First Approach}

\subsection{Step 1: Calculate Z-Coordinate Deltas}
\begin{align}
\DZ{\chainage{w}}{\time{m}}{\Left} &= \Z{\chainage{w}}{\time{m}}{\Left} - \Z{\chainage{w}}{\time{0}}{\Left} \\
\DZ{\chainage{w}}{\time{m}}{\Right} &= \Z{\chainage{w}}{\time{m}}{\Right} - \Z{\chainage{w}}{\time{0}}{\Right} \\
\DZ{\chainage{w} + \Delta ch}{\time{m}}{\Left} &= \Z{\chainage{w} + \Delta ch}{\time{m}}{\Left} - \Z{\chainage{w} + \Delta ch}{\time{0}}{\Left} \\
\DZ{\chainage{w} + \Delta ch}{\time{m}}{\Right} &= \Z{\chainage{w} + \Delta ch}{\time{m}}{\Right} - \Z{\chainage{w} + \Delta ch}{\time{0}}{\Right}
\end{align}

\subsection{Step 2: Calculate Cant Changes}
\begin{align}
\Delta\Cant(\chainage{w}, \time{m}) &= \DZ{\chainage{w}}{\time{m}}{\Left} - \DZ{\chainage{w}}{\time{m}}{\Right} \\
\Delta\Cant(\chainage{w} + \Delta ch, \time{m}) &= \DZ{\chainage{w} + \Delta ch}{\time{m}}{\Left} - \DZ{\chainage{w} + \Delta ch}{\time{m}}{\Right}
\end{align}

\subsection{Step 3: Calculate Twist Change Directly}
\begin{equation}
\DTwist{\chainage{w}}{\Delta ch}{\time{m}}_B = \Delta\Cant(\chainage{w} + \Delta ch, \time{m}) - \Delta\Cant(\chainage{w}, \time{m})
\end{equation}

\section{Equivalence Proof}

\subsection{Expand Method A}
Substitute cant definitions into Method A:
\begin{align}
\DTwist{\chainage{w}}{\Delta ch}{\time{m}}_A &= \left[\left(\Z{\chainage{w} + \Delta ch}{\time{m}}{\Left} - \Z{\chainage{w} + \Delta ch}{\time{m}}{\Right}\right) - \left(\Z{\chainage{w}}{\time{m}}{\Left} - \Z{\chainage{w}}{\time{m}}{\Right}\right)\right] \\
&\quad - \left[\left(\Z{\chainage{w} + \Delta ch}{\time{0}}{\Left} - \Z{\chainage{w} + \Delta ch}{\time{0}}{\Right}\right) - \left(\Z{\chainage{w}}{\time{0}}{\Left} - \Z{\chainage{w}}{\time{0}}{\Right}\right)\right]
\end{align}

Expanding:
\begin{align}
\DTwist{\chainage{w}}{\Delta ch}{\time{m}}_A &= \Z{\chainage{w} + \Delta ch}{\time{m}}{\Left} - \Z{\chainage{w} + \Delta ch}{\time{m}}{\Right} - \Z{\chainage{w}}{\time{m}}{\Left} + \Z{\chainage{w}}{\time{m}}{\Right} \\
&\quad - \Z{\chainage{w} + \Delta ch}{\time{0}}{\Left} + \Z{\chainage{w} + \Delta ch}{\time{0}}{\Right} + \Z{\chainage{w}}{\time{0}}{\Left} - \Z{\chainage{w}}{\time{0}}{\Right}
\end{align}

\subsection{Expand Method B}
Substitute delta definitions into Method B:
\begin{align}
\DTwist{\chainage{w}}{\Delta ch}{\time{m}}_B &= \left[\DZ{\chainage{w} + \Delta ch}{\time{m}}{\Left} - \DZ{\chainage{w} + \Delta ch}{\time{m}}{\Right}\right] - \left[\DZ{\chainage{w}}{\time{m}}{\Left} - \DZ{\chainage{w}}{\time{m}}{\Right}\right]
\end{align}

Substituting delta definitions:
\begin{align}
\DTwist{\chainage{w}}{\Delta ch}{\time{m}}_B &= \left[\Z{\chainage{w} + \Delta ch}{\time{m}}{\Left} - \Z{\chainage{w} + \Delta ch}{\time{0}}{\Left}\right] - \left[\Z{\chainage{w} + \Delta ch}{\time{m}}{\Right} - \Z{\chainage{w} + \Delta ch}{\time{0}}{\Right}\right] \\
&\quad - \left[\Z{\chainage{w}}{\time{m}}{\Left} - \Z{\chainage{w}}{\time{0}}{\Left}\right] + \left[\Z{\chainage{w}}{\time{m}}{\Right} - \Z{\chainage{w}}{\time{0}}{\Right}\right]
\end{align}

Expanding:
\begin{align}
\DTwist{\chainage{w}}{\Delta ch}{\time{m}}_B &= \Z{\chainage{w} + \Delta ch}{\time{m}}{\Left} - \Z{\chainage{w} + \Delta ch}{\time{0}}{\Left} - \Z{\chainage{w} + \Delta ch}{\time{m}}{\Right} + \Z{\chainage{w} + \Delta ch}{\time{0}}{\Right} \\
&\quad - \Z{\chainage{w}}{\time{m}}{\Left} + \Z{\chainage{w}}{\time{0}}{\Left} + \Z{\chainage{w}}{\time{m}}{\Right} - \Z{\chainage{w}}{\time{0}}{\Right}
\end{align}

\subsection{Rearranging Terms}
Rearrange Method A:
\begin{align}
\DTwist{\chainage{w}}{\Delta ch}{\time{m}}_A &= \Z{\chainage{w} + \Delta ch}{\time{m}}{\Left} - \Z{\chainage{w} + \Delta ch}{\time{0}}{\Left} - \Z{\chainage{w} + \Delta ch}{\time{m}}{\Right} + \Z{\chainage{w} + \Delta ch}{\time{0}}{\Right} \\
&\quad - \Z{\chainage{w}}{\time{m}}{\Left} + \Z{\chainage{w}}{\time{0}}{\Left} + \Z{\chainage{w}}{\time{m}}{\Right} - \Z{\chainage{w}}{\time{0}}{\Right}
\end{align}

Compare with Method B - they are identical!

\subsection{Conclusion}
\begin{equation}
\boxed{\DTwist{\chainage{w}}{\Delta ch}{\time{m}}_A = \DTwist{\chainage{w}}{\Delta ch}{\time{m}}_B}
\end{equation}

\textcolor{green}{\textbf{VERIFIED:}} Both methods produce identical results.

\section{General Formula}
The equivalent expressions can be written as:
\begin{equation}
\DTwist{\chainage{w}}{\Delta ch}{\time{m}} = \left[\Delta\Cant(\chainage{w} + \Delta ch, \time{m}) - \Delta\Cant(\chainage{w}, \time{m})\right] = \left[\Twist{\chainage{w}}{\Delta ch}{\time{m}} - \Twist{\chainage{w}}{\Delta ch}{\time{0}}\right]
\end{equation}

\section{Implementation Notes}
\begin{itemize}
\item \textbf{Method A} follows the conceptual definition: calculate twist parameter, then find change
\item \textbf{Method B} is more direct: work with coordinate differences to compute twist change
\item Both methods are mathematically equivalent due to linearity of cant and twist calculations
\item Method B may be computationally more efficient as it avoids intermediate cant calculations
\end{itemize}

\end{document}