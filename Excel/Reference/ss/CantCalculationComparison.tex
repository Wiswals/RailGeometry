\documentclass{article}
\usepackage{amsmath}
\usepackage{amsfonts}
\usepackage{amssymb}
\usepackage{geometry}
\usepackage{xcolor}
\geometry{margin=1in}

% Custom commands for simplified notation
\newcommand{\chainage}[1]{ch_{#1}}
\newcommand{\time}[1]{\tau_{#1}}
\newcommand{\Left}{\text{L}}
\newcommand{\Right}{\text{R}}

% Coordinate commands
\newcommand{\X}[3]{X(#1, #2)_{#3}}
\newcommand{\Y}[3]{Y(#1, #2)_{#3}}
\newcommand{\Z}[3]{Z(#1, #2)_{#3}}

% Delta commands
\newcommand{\DX}[3]{\Delta X(#1, #2)_{#3}}
\newcommand{\DY}[3]{\Delta Y(#1, #2)_{#3}}
\newcommand{\DZ}[3]{\Delta Z(#1, #2)_{#3}}

% Cant commands
\newcommand{\Cant}[2]{\text{Cant}(#1, #2)}
\newcommand{\DCant}[2]{\Delta\text{Cant}(#1, #2)}

\title{Cant Change Calculation Methods - Equivalence Verification}
\author{Rail Geometry Analysis}
\date{}

\begin{document}
\maketitle

\section{Problem Statement}
Given interpolated rail coordinates at chainage $\chainage{c}$ for left and right rails, verify that two methods for calculating cant change produce identical results:

\textbf{Method A}: Calculate cant, then compute change in cant\\
\textbf{Method B}: Calculate coordinate deltas, then compute cant change directly

\section{Input Dataset}
Assume we have interpolated rail coordinates available:
\begin{align}
\text{Left Rail:} \quad &\X{\chainage{c}}{\time{0}}{\Left}, \Y{\chainage{c}}{\time{0}}{\Left}, \Z{\chainage{c}}{\time{0}}{\Left} \quad \text{(baseline)} \\
&\X{\chainage{c}}{\time{m}}{\Left}, \Y{\chainage{c}}{\time{m}}{\Left}, \Z{\chainage{c}}{\time{m}}{\Left} \quad \text{(current)} \\[0.5em]
\text{Right Rail:} \quad &\X{\chainage{c}}{\time{0}}{\Right}, \Y{\chainage{c}}{\time{0}}{\Right}, \Z{\chainage{c}}{\time{0}}{\Right} \quad \text{(baseline)} \\
&\X{\chainage{c}}{\time{m}}{\Right}, \Y{\chainage{c}}{\time{m}}{\Right}, \Z{\chainage{c}}{\time{m}}{\Right} \quad \text{(current)}
\end{align}

\section{Method A: Cant-First Approach}

\subsection{Step 1: Calculate Baseline Cant}
\begin{equation}
\Cant{\chainage{c}}{\time{0}} = \Z{\chainage{c}}{\time{0}}{\Left} - \Z{\chainage{c}}{\time{0}}{\Right}
\end{equation}

\subsection{Step 2: Calculate Current Cant}
\begin{equation}
\Cant{\chainage{c}}{\time{m}} = \Z{\chainage{c}}{\time{m}}{\Left} - \Z{\chainage{c}}{\time{m}}{\Right}
\end{equation}

\subsection{Step 3: Calculate Cant Change}
\begin{equation}
\DCant{\chainage{c}}{\time{m}}_A = \Cant{\chainage{c}}{\time{m}} - \Cant{\chainage{c}}{\time{0}}
\end{equation}

Substituting the cant definitions:
\begin{equation}
\DCant{\chainage{c}}{\time{m}}_A = \left[\Z{\chainage{c}}{\time{m}}{\Left} - \Z{\chainage{c}}{\time{m}}{\Right}\right] - \left[\Z{\chainage{c}}{\time{0}}{\Left} - \Z{\chainage{c}}{\time{0}}{\Right}\right]
\end{equation}

\section{Method B: Delta-First Approach}

\subsection{Step 1: Calculate Z-Coordinate Deltas}
\begin{align}
\DZ{\chainage{c}}{\time{m}}{\Left} &= \Z{\chainage{c}}{\time{m}}{\Left} - \Z{\chainage{c}}{\time{0}}{\Left} \\
\DZ{\chainage{c}}{\time{m}}{\Right} &= \Z{\chainage{c}}{\time{m}}{\Right} - \Z{\chainage{c}}{\time{0}}{\Right}
\end{align}

\subsection{Step 2: Calculate Cant Change Directly}
\begin{equation}
\DCant{\chainage{c}}{\time{m}}_B = \DZ{\chainage{c}}{\time{m}}{\Left} - \DZ{\chainage{c}}{\time{m}}{\Right}
\end{equation}

Substituting the delta definitions:
\begin{equation}
\DCant{\chainage{c}}{\time{m}}_B = \left[\Z{\chainage{c}}{\time{m}}{\Left} - \Z{\chainage{c}}{\time{0}}{\Left}\right] - \left[\Z{\chainage{c}}{\time{m}}{\Right} - \Z{\chainage{c}}{\time{0}}{\Right}\right]
\end{equation}

\section{Equivalence Proof}

\subsection{Algebraic Expansion}
Expand Method A result:
\begin{align}
\DCant{\chainage{c}}{\time{m}}_A &= \left[\Z{\chainage{c}}{\time{m}}{\Left} - \Z{\chainage{c}}{\time{m}}{\Right}\right] - \left[\Z{\chainage{c}}{\time{0}}{\Left} - \Z{\chainage{c}}{\time{0}}{\Right}\right] \\
&= \Z{\chainage{c}}{\time{m}}{\Left} - \Z{\chainage{c}}{\time{m}}{\Right} - \Z{\chainage{c}}{\time{0}}{\Left} + \Z{\chainage{c}}{\time{0}}{\Right}
\end{align}

Expand Method B result:
\begin{align}
\DCant{\chainage{c}}{\time{m}}_B &= \left[\Z{\chainage{c}}{\time{m}}{\Left} - \Z{\chainage{c}}{\time{0}}{\Left}\right] - \left[\Z{\chainage{c}}{\time{m}}{\Right} - \Z{\chainage{c}}{\time{0}}{\Right}\right] \\
&= \Z{\chainage{c}}{\time{m}}{\Left} - \Z{\chainage{c}}{\time{0}}{\Left} - \Z{\chainage{c}}{\time{m}}{\Right} + \Z{\chainage{c}}{\time{0}}{\Right}
\end{align}

\subsection{Rearranging Terms}
Rearrange Method A:
\begin{equation}
\DCant{\chainage{c}}{\time{m}}_A = \Z{\chainage{c}}{\time{m}}{\Left} - \Z{\chainage{c}}{\time{0}}{\Left} - \Z{\chainage{c}}{\time{m}}{\Right} + \Z{\chainage{c}}{\time{0}}{\Right}
\end{equation}

Compare with Method B:
\begin{equation}
\DCant{\chainage{c}}{\time{m}}_B = \Z{\chainage{c}}{\time{m}}{\Left} - \Z{\chainage{c}}{\time{0}}{\Left} - \Z{\chainage{c}}{\time{m}}{\Right} + \Z{\chainage{c}}{\time{0}}{\Right}
\end{equation}

\subsection{Conclusion}
\begin{equation}
\boxed{\DCant{\chainage{c}}{\time{m}}_A = \DCant{\chainage{c}}{\time{m}}_B}
\end{equation}

\textcolor{green}{\textbf{VERIFIED:}} Both methods produce identical results.

\section{General Formula}
The equivalent expressions can be written as:
\begin{equation}
\DCant{\chainage{c}}{\time{m}} = \left[\DZ{\chainage{c}}{\time{m}}{\Left} - \DZ{\chainage{c}}{\time{m}}{\Right}\right] = \left[\Cant{\chainage{c}}{\time{m}} - \Cant{\chainage{c}}{\time{0}}\right]
\end{equation}

\section{Implementation Notes}
\begin{itemize}
\item \textbf{Method A} is conceptually clearer: calculate geometry parameter, then find change
\item \textbf{Method B} is computationally efficient: work directly with coordinate differences
\item Both methods are mathematically equivalent due to linearity of the cant calculation
\item Choose based on implementation preference or computational efficiency requirements
\end{itemize}

\section{Extension to Twist Calculations}
Since twist is the difference in cant over a distance interval, the same equivalence principle applies.

\subsection{Twist Definition}
For chainages $\chainage{w}$ and $\chainage{w} + \Delta ch$:
\begin{equation}
\Twist(\chainage{w}, \Delta ch, \time{m}) = \Cant(\chainage{w} + \Delta ch, \time{m}) - \Cant(\chainage{w}, \time{m})
\end{equation}

\subsection{Twist Change Methods}
\textbf{Method A (Twist-First):}
\begin{equation}
\DTwist{\chainage{w}}{\Delta ch}{\time{m}}_A = \Twist(\chainage{w}, \Delta ch, \time{m}) - \Twist(\chainage{w}, \Delta ch, \time{0})
\end{equation}

\textbf{Method B (Delta-First):}
\begin{equation}
\DTwist{\chainage{w}}{\Delta ch}{\time{m}}_B = [\Delta\Cant(\chainage{w} + \Delta ch, \time{m}) - \Delta\Cant(\chainage{w}, \time{m})]
\end{equation}

where $\Delta\Cant$ values are calculated using the cant change methods above.

\subsection{Twist Equivalence}
Since twist is a linear combination of cant values, and cant change calculations are equivalent:
\begin{equation}
\boxed{\DTwist{\chainage{w}}{\Delta ch}{\time{m}}_A = \DTwist{\chainage{w}}{\Delta ch}{\time{m}}_B}
\end{equation}

\textcolor{green}{\textbf{VERIFIED:}} Twist change methods are equivalent by transitivity of the cant equivalence.

\subsection{Computational Efficiency}
\textbf{Method B for Twist} is particularly efficient as it:
\begin{itemize}
\item Reuses the $\Delta Z$ calculations from cant analysis
\item Avoids computing intermediate cant and twist values
\item Directly computes twist change from coordinate deltas
\end{itemize}

\section{Summary}
Both cant and twist change calculations demonstrate mathematical equivalence between parameter-first and delta-first approaches. The choice depends on computational efficiency needs and conceptual clarity preferences.

\end{document}