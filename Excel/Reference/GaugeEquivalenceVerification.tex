\documentclass{article}
\usepackage{amsmath}
\usepackage{amsfonts}
\usepackage{amssymb}
\usepackage{geometry}
\usepackage{xcolor}
\geometry{margin=1in}

\title{Rail Geometry Gauge Change Calculations - Equivalence Verification}
\author{Rail Geometry Analysis}
\date{}

\begin{document}
\maketitle

\section{Problem Statement}
Given interpolated rail coordinates for left and right rails, verify that two computational approaches for calculating gauge parameter changes produce identical results:

\textbf{Parameter-First Approach}: Calculate gauge parameters, then compute parameter changes\\
\textbf{Delta-First Approach}: Calculate coordinate deltas, then compute parameter changes directly

This verification demonstrates mathematical equivalence between both computational methods for gauge calculations.

\section{Notation}
\textbf{Variable Definitions:}
\begin{itemize}
\item $ch_c$ = Chainage location (distance along track centerline) where gauge is calculated
\item $\tau_0$ = Baseline time (reference measurement)
\item $\tau_m$ = Current measurement time (m = 1, 2, 3, ...)
\item $\text{L}$ = Left rail (subscript)
\item $\text{R}$ = Right rail (subscript)
\item $X, Y, Z$ = 3D coordinates (X = Easting, Y = Northing, Z = Elevation)
\item $\Delta$ = Change or difference operator
\end{itemize}

\textbf{Coordinate Notation:}
\begin{itemize}
\item $X(ch_c, \tau_m)_L$ = X-coordinate of left rail at chainage $ch_c$ and time $\tau_m$
\item $Y(ch_c, \tau_m)_L$ = Y-coordinate of left rail at chainage $ch_c$ and time $\tau_m$
\item $Z(ch_c, \tau_m)_L$ = Z-coordinate of left rail at chainage $ch_c$ and time $\tau_m$
\item $\Delta X(ch_c, \tau_m)_L$ = Change in X-coordinate: $X(ch_c, \tau_m)_L - X(ch_c, \tau_0)_L$
\end{itemize}

\textbf{Parameter Notation:}
\begin{itemize}
\item $\text{Gauge}(ch_c, \tau_m)$ = Rail gauge at chainage $ch_c$ and time $\tau_m$
\item $\Delta\text{Gauge}(ch_c, \tau_m)$ = Change in gauge from baseline to time $\tau_m$
\end{itemize}

\section{Input Dataset}
Assume we have interpolated rail coordinates available at a specific chainage location:
\begin{align}
\text{Left Rail:} \quad &X(ch_c, \tau_0)_L, Y(ch_c, \tau_0)_L, Z(ch_c, \tau_0)_L \quad \text{(baseline)} \\
&X(ch_c, \tau_m)_L, Y(ch_c, \tau_m)_L, Z(ch_c, \tau_m)_L \quad \text{(current)} \\[0.5em]
\text{Right Rail:} \quad &X(ch_c, \tau_0)_R, Y(ch_c, \tau_0)_R, Z(ch_c, \tau_0)_R \quad \text{(baseline)} \\
&X(ch_c, \tau_m)_R, Y(ch_c, \tau_m)_R, Z(ch_c, \tau_m)_R \quad \text{(current)}
\end{align}

\textbf{Example:} At chainage 1000m, we have baseline and current 3D coordinates for both left and right rails.

\section{Parameter-First Approach}

\subsection{Step 1: Calculate Baseline Gauge}
For 3D gauge (true spatial distance):
\begin{equation}
\text{Gauge}(ch_c, \tau_0) = \sqrt{(X(ch_c, \tau_0)_L - X(ch_c, \tau_0)_R)^2 + (Y(ch_c, \tau_0)_L - Y(ch_c, \tau_0)_R)^2 + (Z(ch_c, \tau_0)_L - Z(ch_c, \tau_0)_R)^2}
\end{equation}

\subsection{Step 2: Calculate Current Gauge}
\begin{equation}
\text{Gauge}(ch_c, \tau_m) = \sqrt{(X(ch_c, \tau_m)_L - X(ch_c, \tau_m)_R)^2 + (Y(ch_c, \tau_m)_L - Y(ch_c, \tau_m)_R)^2 + (Z(ch_c, \tau_m)_L - Z(ch_c, \tau_m)_R)^2}
\end{equation}

\subsection{Step 3: Calculate Gauge Change}
\begin{equation}
\Delta\text{Gauge}(ch_c, \tau_m)_{\text{param}} = \text{Gauge}(ch_c, \tau_m) - \text{Gauge}(ch_c, \tau_0)
\end{equation}

Substituting the gauge definitions:
\begin{align}
\Delta\text{Gauge}(ch_c, \tau_m)_{\text{param}} &= \sqrt{(X(ch_c, \tau_m)_L - X(ch_c, \tau_m)_R)^2 + (Y(ch_c, \tau_m)_L - Y(ch_c, \tau_m)_R)^2 + (Z(ch_c, \tau_m)_L - Z(ch_c, \tau_m)_R)^2} \nonumber \\
&\quad - \sqrt{(X(ch_c, \tau_0)_L - X(ch_c, \tau_0)_R)^2 + (Y(ch_c, \tau_0)_L - Y(ch_c, \tau_0)_R)^2 + (Z(ch_c, \tau_0)_L - Z(ch_c, \tau_0)_R)^2}
\end{align}

\section{Delta-First Approach}

\subsection{Step 1: Calculate Coordinate Deltas}
\begin{align}
\Delta X(ch_c, \tau_m)_L &= X(ch_c, \tau_m)_L - X(ch_c, \tau_0)_L \\
\Delta Y(ch_c, \tau_m)_L &= Y(ch_c, \tau_m)_L - Y(ch_c, \tau_0)_L \\
\Delta Z(ch_c, \tau_m)_L &= Z(ch_c, \tau_m)_L - Z(ch_c, \tau_0)_L \\
\Delta X(ch_c, \tau_m)_R &= X(ch_c, \tau_m)_R - X(ch_c, \tau_0)_R \\
\Delta Y(ch_c, \tau_m)_R &= Y(ch_c, \tau_m)_R - Y(ch_c, \tau_0)_R \\
\Delta Z(ch_c, \tau_m)_R &= Z(ch_c, \tau_m)_R - Z(ch_c, \tau_0)_R
\end{align}

\subsection{Step 2: Define Separation Vectors}
Define baseline separation vector:
\begin{equation}
\vec{S}_0 = (X(ch_c, \tau_0)_L - X(ch_c, \tau_0)_R, Y(ch_c, \tau_0)_L - Y(ch_c, \tau_0)_R, Z(ch_c, \tau_0)_L - Z(ch_c, \tau_0)_R)
\end{equation}

Define delta separation vector:
\begin{align}
\vec{\Delta S} &= (\Delta X(ch_c, \tau_m)_L - \Delta X(ch_c, \tau_m)_R, \nonumber \\
&\quad \Delta Y(ch_c, \tau_m)_L - \Delta Y(ch_c, \tau_m)_R, \nonumber \\
&\quad \Delta Z(ch_c, \tau_m)_L - \Delta Z(ch_c, \tau_m)_R)
\end{align}

\subsection{Step 3: Calculate Gauge Change Using Vector Addition}
\begin{equation}
\Delta\text{Gauge}(ch_c, \tau_m)_{\text{delta}} = |\vec{S}_0 + \vec{\Delta S}| - |\vec{S}_0|
\end{equation}

\section{Equivalence Proof}

\subsection{Expand Current Separation Vector}
The current separation vector is:
\begin{equation}
\vec{S}_0 + \vec{\Delta S} = (X(ch_c, \tau_m)_L - X(ch_c, \tau_m)_R, Y(ch_c, \tau_m)_L - Y(ch_c, \tau_m)_R, Z(ch_c, \tau_m)_L - Z(ch_c, \tau_m)_R)
\end{equation}

\subsection{Expand Delta-First Method}
Substituting vector definitions:
\begin{align}
\Delta\text{Gauge}(ch_c, \tau_m)_{\text{delta}} &= |\vec{S}_0 + \vec{\Delta S}| - |\vec{S}_0| \nonumber \\
&= \sqrt{(X(ch_c, \tau_m)_L - X(ch_c, \tau_m)_R)^2 + (Y(ch_c, \tau_m)_L - Y(ch_c, \tau_m)_R)^2 + (Z(ch_c, \tau_m)_L - Z(ch_c, \tau_m)_R)^2} \nonumber \\
&\quad - \sqrt{(X(ch_c, \tau_0)_L - X(ch_c, \tau_0)_R)^2 + (Y(ch_c, \tau_0)_L - Y(ch_c, \tau_0)_R)^2 + (Z(ch_c, \tau_0)_L - Z(ch_c, \tau_0)_R)^2}
\end{align}

\subsection{Final Comparison}
Both methods yield identical expressions:
\begin{align}
\Delta\text{Gauge}(ch_c, \tau_m)_{\text{param}} &= \sqrt{(X(ch_c, \tau_m)_L - X(ch_c, \tau_m)_R)^2 + (Y(ch_c, \tau_m)_L - Y(ch_c, \tau_m)_R)^2 + (Z(ch_c, \tau_m)_L - Z(ch_c, \tau_m)_R)^2} \nonumber \\
&\quad - \sqrt{(X(ch_c, \tau_0)_L - X(ch_c, \tau_0)_R)^2 + (Y(ch_c, \tau_0)_L - Y(ch_c, \tau_0)_R)^2 + (Z(ch_c, \tau_0)_L - Z(ch_c, \tau_0)_R)^2} \\
\Delta\text{Gauge}(ch_c, \tau_m)_{\text{delta}} &= \sqrt{(X(ch_c, \tau_m)_L - X(ch_c, \tau_m)_R)^2 + (Y(ch_c, \tau_m)_L - Y(ch_c, \tau_m)_R)^2 + (Z(ch_c, \tau_m)_L - Z(ch_c, \tau_m)_R)^2} \nonumber \\
&\quad - \sqrt{(X(ch_c, \tau_0)_L - X(ch_c, \tau_0)_R)^2 + (Y(ch_c, \tau_0)_L - Y(ch_c, \tau_0)_R)^2 + (Z(ch_c, \tau_0)_L - Z(ch_c, \tau_0)_R)^2}
\end{align}

\section{Worked Example}
Consider a specific numerical example to demonstrate both computational methods.

\subsection{Given Data}
At chainage $ch_c = 1000$m:
\begin{align}
\text{Baseline (}\tau_0\text{):} \quad &X(1000, \tau_0)_L = 500.125\text{m}, \quad Y(1000, \tau_0)_L = 200.345\text{m}, \quad Z(1000, \tau_0)_L = 102.345\text{m} \nonumber \\
&X(1000, \tau_0)_R = 498.590\text{m}, \quad Y(1000, \tau_0)_R = 200.355\text{m}, \quad Z(1000, \tau_0)_R = 102.330\text{m} \\[0.5em]
\text{Current (}\tau_m\text{):} \quad &X(1000, \tau_m)_L = 500.120\text{m}, \quad Y(1000, \tau_m)_L = 200.350\text{m}, \quad Z(1000, \tau_m)_L = 102.358\text{m} \nonumber \\
&X(1000, \tau_m)_R = 498.598\text{m}, \quad Y(1000, \tau_m)_R = 200.352\text{m}, \quad Z(1000, \tau_m)_R = 102.340\text{m}
\end{align}

\subsection{Parameter-First Calculation}
\textbf{Step 1:} Calculate baseline gauge
\begin{align}
\text{Gauge}(1000, \tau_0) &= \sqrt{(500.125 - 498.590)^2 + (200.345 - 200.355)^2 + (102.345 - 102.330)^2} \nonumber \\
&= \sqrt{(1.535)^2 + (-0.010)^2 + (0.015)^2} \nonumber \\
&= \sqrt{2.356225 + 0.000100 + 0.000225} \nonumber \\
&= \sqrt{2.356550} = 1.5351\text{m}
\end{align}

\textbf{Step 2:} Calculate current gauge
\begin{align}
\text{Gauge}(1000, \tau_m) &= \sqrt{(500.120 - 498.598)^2 + (200.350 - 200.352)^2 + (102.358 - 102.340)^2} \nonumber \\
&= \sqrt{(1.522)^2 + (-0.002)^2 + (0.018)^2} \nonumber \\
&= \sqrt{2.316484 + 0.000004 + 0.000324} \nonumber \\
&= \sqrt{2.316812} = 1.5222\text{m}
\end{align}

\textbf{Step 3:} Calculate gauge change
\begin{equation}
\Delta\text{Gauge}(1000, \tau_m)_{\text{param}} = 1.5222 - 1.5351 = -0.0129\text{m}
\end{equation}

\subsection{Delta-First Calculation}
\textbf{Step 1:} Calculate coordinate deltas
\begin{align}
\Delta X(1000, \tau_m)_L &= 500.120 - 500.125 = -0.005\text{m} \\
\Delta Y(1000, \tau_m)_L &= 200.350 - 200.345 = 0.005\text{m} \\
\Delta Z(1000, \tau_m)_L &= 102.358 - 102.345 = 0.013\text{m} \\
\Delta X(1000, \tau_m)_R &= 498.598 - 498.590 = 0.008\text{m} \\
\Delta Y(1000, \tau_m)_R &= 200.352 - 200.355 = -0.003\text{m} \\
\Delta Z(1000, \tau_m)_R &= 102.340 - 102.330 = 0.010\text{m}
\end{align}

\textbf{Step 2:} Define separation vectors

Baseline separation vector (left rail - right rail at baseline):
\begin{align}
\vec{S}_0 &= (X(1000, \tau_0)_L - X(1000, \tau_0)_R, Y(1000, \tau_0)_L - Y(1000, \tau_0)_R, Z(1000, \tau_0)_L - Z(1000, \tau_0)_R) \nonumber \\
&= (500.125 - 498.590, 200.345 - 200.355, 102.345 - 102.330) \nonumber \\
&= (1.535, -0.010, 0.015)
\end{align}

Delta separation vector (difference in coordinate deltas):
\begin{align}
\vec{\Delta S} &= (\Delta X(1000, \tau_m)_L - \Delta X(1000, \tau_m)_R, \Delta Y(1000, \tau_m)_L - \Delta Y(1000, \tau_m)_R, \Delta Z(1000, \tau_m)_L - \Delta Z(1000, \tau_m)_R) \nonumber \\
&= ((-0.005) - (0.008), (0.005) - (-0.003), (0.013) - (0.010)) \nonumber \\
&= (-0.013, 0.008, 0.003)
\end{align}

\textbf{Step 3:} Calculate gauge change using vector addition
\begin{align}
\vec{S}_0 + \vec{\Delta S} &= (1.535 - 0.013, -0.010 + 0.008, 0.015 + 0.003) = (1.522, -0.002, 0.018) \\
|\vec{S}_0 + \vec{\Delta S}| &= \sqrt{(1.522)^2 + (-0.002)^2 + (0.018)^2} = 1.5222\text{m} \\
\Delta\text{Gauge}(1000, \tau_m)_{\text{delta}} &= 1.5222 - 1.5351 = -0.0129\text{m}
\end{align}

\subsection{Verification}
Both methods yield identical results:
\begin{equation}
\Delta\text{Gauge}(1000, \tau_m)_{\text{param}} = \Delta\text{Gauge}(1000, \tau_m)_{\text{delta}} = -0.0129\text{m}
\end{equation}

\section{Conclusion}
\begin{equation}
\boxed{\Delta\text{Gauge}(ch_c, \tau_m)_{\text{param}} = \Delta\text{Gauge}(ch_c, \tau_m)_{\text{delta}}}
\end{equation}

\textcolor{green}{\textbf{VERIFIED:}} Both approaches produce identical results for gauge change calculations.

\textbf{Mathematical Basis:} The equivalence holds because the vector addition $\vec{S}_0 + \vec{\Delta S}$ produces the same current separation vector used in the parameter-first approach, ensuring identical distance calculations.

\section{Implementation Notes}
\begin{itemize}
\item \textbf{Parameter-First}: More intuitive, provides intermediate gauge values for analysis
\item \textbf{Delta-First}: More computationally efficient, reuses coordinate deltas across multiple parameters
\item Both methods are mathematically equivalent and produce identical numerical results
\item Gauge calculations involve 3D distance computations using all coordinate components
\item The vector approach in delta-first method provides clear geometric interpretation
\item Choice depends on computational efficiency needs and whether intermediate gauge values are required
\end{itemize}

\end{document}on
\begin{align}
\vec{S}_0 + \vec{\Delta S} &= (1.535, -0.010, 0.018) \\
|\vec{S}_0 + \vec{\Delta S}| &= \sqrt{(1.535)^2 + (-0.010)^2 + (0.018)^2} = 1.535\text{m} \\
\Delta\text{Gauge}(1000, \tau_m)_{\text{delta}} &= 1.535 - 1.535 = 0.000\text{m}
\end{align}

\subsection{Verification}
Both methods yield identical results:
\begin{equation}
\Delta\text{Gauge}(1000, \tau_m)_{\text{param}} = \Delta\text{Gauge}(1000, \tau_m)_{\text{delta}} = 0.000\text{m}
\end{equation}

\section{Conclusion}
\begin{equation}
\boxed{\Delta\text{Gauge}(ch_c, \tau_m)_{\text{param}} = \Delta\text{Gauge}(ch_c, \tau_m)_{\text{delta}}}
\end{equation}

\textcolor{green}{\textbf{VERIFIED:}} Both approaches produce identical results for gauge change calculations.

\textbf{Mathematical Basis:} The equivalence holds because the vector addition $\vec{S}_0 + \vec{\Delta S}$ produces the same current separation vector used in the parameter-first approach, ensuring identical distance calculations.

\section{Implementation Notes}
\begin{itemize}
\item \textbf{Parameter-First}: More intuitive, provides intermediate gauge values for analysis
\item \textbf{Delta-First}: More computationally efficient, reuses coordinate deltas across multiple parameters
\item Both methods are mathematically equivalent and produce identical numerical results
\item Gauge calculations involve 3D distance computations using all coordinate components
\item The vector approach in delta-first method provides clear geometric interpretation
\item Choice depends on computational efficiency needs and whether intermediate gauge values are required
\end{itemize}

\end{document}