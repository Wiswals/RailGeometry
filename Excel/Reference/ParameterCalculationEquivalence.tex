\documentclass{article}
\usepackage{amsmath}
\usepackage{amsfonts}
\usepackage{amssymb}
\usepackage{geometry}
\usepackage{xcolor}
\geometry{margin=1in}

% Custom commands for simplified notation
\newcommand{\chainage}[1]{ch_{#1}}
\newcommand{\Left}{\text{L}}
\newcommand{\Right}{\text{R}}

% Coordinate commands with explicit tau notation
\newcommand{\X}[3]{X(#1, #2)_{#3}}
\newcommand{\Y}[3]{Y(#1, #2)_{#3}}
\newcommand{\Z}[3]{Z(#1, #2)_{#3}}

% Delta commands with explicit tau notation
\newcommand{\DX}[3]{\Delta X(#1, #2)_{#3}}
\newcommand{\DY}[3]{\Delta Y(#1, #2)_{#3}}
\newcommand{\DZ}[3]{\Delta Z(#1, #2)_{#3}}

% Geometry parameter commands with explicit tau notation
\newcommand{\Cant}[2]{\text{Cant}(#1, #2)}
\newcommand{\DCant}[2]{\Delta\text{Cant}(#1, #2)}
\newcommand{\Gauge}[2]{\text{Gauge}(#1, #2)}
\newcommand{\DGauge}[2]{\Delta\text{Gauge}(#1, #2)}
\newcommand{\Twist}[3]{\text{Twist}(#1, #2, #3)}
\newcommand{\DTwist}[3]{\Delta\text{Twist}(#1, #2, #3)}

% Separation component commands for gauge calculations
\newcommand{\SepX}[2]{\Delta X_{\text{sep}}(#1, #2)}
\newcommand{\SepY}[2]{\Delta Y_{\text{sep}}(#1, #2)}
\newcommand{\SepZ}[2]{\Delta Z_{\text{sep}}(#1, #2)}

% Versine commands
\newcommand{\Versine}[2]{\text{Versine}(#1, #2)}
\newcommand{\DVersine}[2]{\Delta\text{Versine}(#1, #2)}

\title{Rail Geometry Parameter Change Calculations - Equivalence Verification}
\author{Rail Geometry Analysis}
\date{}

\begin{document}
\maketitle

\section{Problem Statement}
Given interpolated rail coordinates for left and right rails, verify that two computational approaches for calculating geometry parameter changes produce identical results:

\textbf{Parameter-First Approach}: Calculate geometry parameters, then compute parameter changes\\
\textbf{Delta-First Approach}: Calculate coordinate deltas, then compute parameter changes directly

This verification covers cant, gauge, twist, and horizontal versine parameters, demonstrating mathematical equivalence between both computational methods.

\section{Notation}
\textbf{Variable Definitions:}
\begin{itemize}
\item $ch_c$ = Chainage location (distance along track centerline) where cant/gauge are calculated
\item $ch_w$ = Chainage location where twist is calculated
\item $\tau_0$ = Baseline time (reference measurement)
\item $\tau_m$ = Current measurement time (m = 1, 2, 3, ...)
\item $\text{L}$ = Left rail (subscript)
\item $\text{R}$ = Right rail (subscript)
\item $X, Y, Z$ = 3D coordinates (X = Easting, Y = Northing, Z = Elevation)
\item $\Delta$ = Change or difference operator
\end{itemize}

\textbf{Coordinate Notation:}
\begin{itemize}
\item $X(ch_c, \tau_m)_L$ = X-coordinate of left rail at chainage $ch_c$ and time $\tau_m$
\item $\Delta X(ch_c, \tau_m)_L$ = Change in X-coordinate: $X(ch_c, \tau_m)_L - X(ch_c, \tau_0)_L$
\end{itemize}

\textbf{Parameter Notation:}
\begin{itemize}
\item $\text{Cant}(ch_c, \tau_m)$ = Cant (cross-level) at chainage $ch_c$ and time $\tau_m$
\item $\Delta\text{Cant}(ch_c, \tau_m)$ = Change in cant from baseline to time $\tau_m$
\item $\text{Gauge}(ch_c, \tau_m)$ = Rail gauge at chainage $ch_c$ and time $\tau_m$
\item $\text{Twist}(ch_w, \Delta ch, \tau_m)$ = Twist over interval $\Delta ch$ starting at chainage $ch_w$
\item $\text{Versine}(ch_v, \tau_m)$ = Horizontal versine at chainage $ch_v$ and time $\tau_m$
\end{itemize}

\section{Input Dataset}
Assume we have interpolated rail coordinates available at a specific chainage location:
\begin{align}
\text{Left Rail:} \quad &\X{\chainage{c}}{\tau_0}{\Left}, \Y{\chainage{c}}{\tau_0}{\Left}, \Z{\chainage{c}}{\tau_0}{\Left} \quad \text{(baseline measurement)} \\
&\X{\chainage{c}}{\tau_m}{\Left}, \Y{\chainage{c}}{\tau_m}{\Left}, \Z{\chainage{c}}{\tau_m}{\Left} \quad \text{(current measurement)} \\[0.5em]
\text{Right Rail:} \quad &\X{\chainage{c}}{\tau_0}{\Right}, \Y{\chainage{c}}{\tau_0}{\Right}, \Z{\chainage{c}}{\tau_0}{\Right} \quad \text{(baseline measurement)} \\
&\X{\chainage{c}}{\tau_m}{\Right}, \Y{\chainage{c}}{\tau_m}{\Right}, \Z{\chainage{c}}{\tau_m}{\Right} \quad \text{(current measurement)}
\end{align}

\textbf{Example:} At chainage 1000m, we have baseline coordinates from January 2024 ($\tau_0$) and current coordinates from June 2024 ($\tau_m$) for both left and right rails.

\section{Cant Change Calculations}

\subsection{Parameter-First Approach}

\subsubsection{Step 1: Calculate Baseline Cant}
\begin{equation}
\Cant{\chainage{c}}{\tau_0} = \Z{\chainage{c}}{\tau_0}{\Left} - \Z{\chainage{c}}{\tau_0}{\Right}
\end{equation}

\subsubsection{Step 2: Calculate Current Cant}
\begin{equation}
\Cant{\chainage{c}}{\tau_m} = \Z{\chainage{c}}{\tau_m}{\Left} - \Z{\chainage{c}}{\tau_m}{\Right}
\end{equation}

\subsubsection{Step 3: Calculate Cant Change}
\begin{equation}
\DCant{\chainage{c}}{\tau_m}_{\text{param}} = \Cant{\chainage{c}}{\tau_m} - \Cant{\chainage{c}}{\tau_0}
\end{equation}

Substituting the cant definitions:
\begin{multline}
\DCant{\chainage{c}}{\tau_m}_{\text{param}} = \left[\Z{\chainage{c}}{\tau_m}{\Left} - \Z{\chainage{c}}{\tau_m}{\Right}\right] \\
- \left[\Z{\chainage{c}}{\tau_0}{\Left} - \Z{\chainage{c}}{\tau_0}{\Right}\right]
\end{multline}

\subsection{Delta-First Approach}

\subsubsection{Step 1: Calculate Z-Coordinate Deltas}
\begin{align}
\DZ{\chainage{c}}{\tau_m}{\Left} &= \Z{\chainage{c}}{\tau_m}{\Left} - \Z{\chainage{c}}{\tau_0}{\Left} \\
\DZ{\chainage{c}}{\tau_m}{\Right} &= \Z{\chainage{c}}{\tau_m}{\Right} - \Z{\chainage{c}}{\tau_0}{\Right}
\end{align}

\subsubsection{Step 2: Calculate Cant Change Directly}
\begin{equation}
\DCant{\chainage{c}}{\tau_m}_{\text{delta}} = \DZ{\chainage{c}}{\tau_m}{\Left} - \DZ{\chainage{c}}{\tau_m}{\Right}
\end{equation}

Substituting the delta definitions:
\begin{multline}
\DCant{\chainage{c}}{\tau_m}_{\text{delta}} = \left[\Z{\chainage{c}}{\tau_m}{\Left} - \Z{\chainage{c}}{\tau_0}{\Left}\right] \\
- \left[\Z{\chainage{c}}{\tau_m}{\Right} - \Z{\chainage{c}}{\tau_0}{\Right}\right]
\end{multline}

\subsection{Cant Equivalence Proof}

\subsubsection{Algebraic Expansion}
Expand parameter-first result:
\begin{align}
\DCant{\chainage{c}}{\tau_m}_{\text{param}} &= \left[\Z{\chainage{c}}{\tau_m}{\Left} - \Z{\chainage{c}}{\tau_m}{\Right}\right] \\
&\quad - \left[\Z{\chainage{c}}{\tau_0}{\Left} - \Z{\chainage{c}}{\tau_0}{\Right}\right] \\
&= \Z{\chainage{c}}{\tau_m}{\Left} - \Z{\chainage{c}}{\tau_m}{\Right} \\
&\quad - \Z{\chainage{c}}{\tau_0}{\Left} + \Z{\chainage{c}}{\tau_0}{\Right}
\end{align}

Expand delta-first result:
\begin{align}
\DCant{\chainage{c}}{\tau_m}_{\text{delta}} &= \left[\Z{\chainage{c}}{\tau_m}{\Left} - \Z{\chainage{c}}{\tau_0}{\Left}\right] \\
&\quad - \left[\Z{\chainage{c}}{\tau_m}{\Right} - \Z{\chainage{c}}{\tau_0}{\Right}\right] \\
&= \Z{\chainage{c}}{\tau_m}{\Left} - \Z{\chainage{c}}{\tau_0}{\Left} \\
&\quad - \Z{\chainage{c}}{\tau_m}{\Right} + \Z{\chainage{c}}{\tau_0}{\Right}
\end{align}

\subsubsection{Rearranging Terms}
Rearrange parameter-first:
\begin{multline}
\DCant{\chainage{c}}{\tau_m}_{\text{param}} = \Z{\chainage{c}}{\tau_m}{\Left} - \Z{\chainage{c}}{\tau_0}{\Left} \\
- \Z{\chainage{c}}{\tau_m}{\Right} + \Z{\chainage{c}}{\tau_0}{\Right}
\end{multline}

Compare with delta-first:
\begin{multline}
\DCant{\chainage{c}}{\tau_m}_{\text{delta}} = \Z{\chainage{c}}{\tau_m}{\Left} - \Z{\chainage{c}}{\tau_0}{\Left} \\
- \Z{\chainage{c}}{\tau_m}{\Right} + \Z{\chainage{c}}{\tau_0}{\Right}
\end{multline}

\subsubsection{Conclusion}
\begin{equation}
\boxed{\DCant{\chainage{c}}{\tau_m}_{\text{param}} = \DCant{\chainage{c}}{\tau_m}_{\text{delta}}}
\end{equation}

\textcolor{green}{\textbf{VERIFIED:}} Both approaches produce identical results for cant change.

\section{Gauge Change Calculations}
Gauge uses the same coordinate inputs as cant but involves distance calculations rather than simple differences.

\subsection{Gauge Definition}
For 3D gauge (true spatial distance), first define separation components:
\begin{align}
\SepX{\chainage{c}}{\tau_m} &= \X{\chainage{c}}{\tau_m}{\Left} - \X{\chainage{c}}{\tau_m}{\Right} \\
\SepY{\chainage{c}}{\tau_m} &= \Y{\chainage{c}}{\tau_m}{\Left} - \Y{\chainage{c}}{\tau_m}{\Right} \\
\SepZ{\chainage{c}}{\tau_m} &= \Z{\chainage{c}}{\tau_m}{\Left} - \Z{\chainage{c}}{\tau_m}{\Right}
\end{align}

Then gauge becomes:
\begin{equation}
\Gauge{\chainage{c}}{\tau_m} = \sqrt{\SepX{\chainage{c}}{\tau_m}^2 + \SepY{\chainage{c}}{\tau_m}^2 + \SepZ{\chainage{c}}{\tau_m}^2}
\end{equation}

\subsection{Parameter-First Approach}

\subsubsection{Step 1: Calculate Baseline Gauge}
\begin{equation}
\Gauge{\chainage{c}}{\tau_0} = \sqrt{\SepX{\chainage{c}}{\tau_0}^2 + \SepY{\chainage{c}}{\tau_0}^2 + \SepZ{\chainage{c}}{\tau_0}^2}
\end{equation}

\subsubsection{Step 2: Calculate Current Gauge}
\begin{equation}
\Gauge{\chainage{c}}{\tau_m} = \sqrt{\SepX{\chainage{c}}{\tau_m}^2 + \SepY{\chainage{c}}{\tau_m}^2 + \SepZ{\chainage{c}}{\tau_m}^2}
\end{equation}

\subsubsection{Step 3: Calculate Gauge Change}
\begin{equation}
\DGauge{\chainage{c}}{\tau_m}_{\text{param}} = \Gauge{\chainage{c}}{\tau_m} - \Gauge{\chainage{c}}{\tau_0}
\end{equation}

\subsection{Delta-First Approach}

\subsubsection{Step 1: Calculate Coordinate Deltas}
Using the same coordinate deltas as calculated for cant:
\begin{align}
\DX{\chainage{c}}{\tau_m}{\Left} &= \X{\chainage{c}}{\tau_m}{\Left} - \X{\chainage{c}}{\tau_0}{\Left} \\
\DY{\chainage{c}}{\tau_m}{\Left} &= \Y{\chainage{c}}{\tau_m}{\Left} - \Y{\chainage{c}}{\tau_0}{\Left} \\
\DZ{\chainage{c}}{\tau_m}{\Left} &= \Z{\chainage{c}}{\tau_m}{\Left} - \Z{\chainage{c}}{\tau_0}{\Left}
\end{align}
\begin{align}
\DX{\chainage{c}}{\tau_m}{\Right} &= \X{\chainage{c}}{\tau_m}{\Right} - \X{\chainage{c}}{\tau_0}{\Right} \\
\DY{\chainage{c}}{\tau_m}{\Right} &= \Y{\chainage{c}}{\tau_m}{\Right} - \Y{\chainage{c}}{\tau_0}{\Right} \\
\DZ{\chainage{c}}{\tau_m}{\Right} &= \Z{\chainage{c}}{\tau_m}{\Right} - \Z{\chainage{c}}{\tau_0}{\Right}
\end{align}

\subsubsection{Step 2: Calculate Gauge Change Using Vector Addition}
Define baseline and delta separation vectors:
\begin{equation}
\vec{S}_0 = (\SepX{\chainage{c}}{\tau_0}, \SepY{\chainage{c}}{\tau_0}, \SepZ{\chainage{c}}{\tau_0})
\end{equation}
\begin{multline}
\vec{\Delta S} = (\DX{\chainage{c}}{\tau_m}{\Left} - \DX{\chainage{c}}{\tau_m}{\Right}, \\
\DY{\chainage{c}}{\tau_m}{\Left} - \DY{\chainage{c}}{\tau_m}{\Right}, \\
\DZ{\chainage{c}}{\tau_m}{\Left} - \DZ{\chainage{c}}{\tau_m}{\Right})
\end{multline}

Then:
\begin{equation}
\DGauge{\chainage{c}}{\tau_m}_{\text{delta}} = |\vec{S}_0 + \vec{\Delta S}| - |\vec{S}_0|
\end{equation}

\subsection{Gauge Equivalence Proof}

\subsubsection{Expand Parameter-First Method}
Substitute gauge definitions:
\begin{align}
\DGauge{\chainage{c}}{\tau_m}_{\text{param}} &= \Gauge{\chainage{c}}{\tau_m} - \Gauge{\chainage{c}}{\tau_0} \\
&= \sqrt{\SepX{\chainage{c}}{\tau_m}^2 + \SepY{\chainage{c}}{\tau_m}^2 + \SepZ{\chainage{c}}{\tau_m}^2} \\
&\quad - \sqrt{\SepX{\chainage{c}}{\tau_0}^2 + \SepY{\chainage{c}}{\tau_0}^2 + \SepZ{\chainage{c}}{\tau_0}^2}
\end{align}

\subsubsection{Expand Delta-First Method}
The current separation vector becomes:
\begin{equation}
\vec{S}_0 + \vec{\Delta S} = (\SepX{\chainage{c}}{\tau_m}, \SepY{\chainage{c}}{\tau_m}, \SepZ{\chainage{c}}{\tau_m})
\end{equation}

Therefore:
\begin{align}
\DGauge{\chainage{c}}{\tau_m}_{\text{delta}} &= |\vec{S}_0 + \vec{\Delta S}| - |\vec{S}_0| \\
&= \sqrt{\SepX{\chainage{c}}{\tau_m}^2 + \SepY{\chainage{c}}{\tau_m}^2 + \SepZ{\chainage{c}}{\tau_m}^2} \\
&\quad - \sqrt{\SepX{\chainage{c}}{\tau_0}^2 + \SepY{\chainage{c}}{\tau_0}^2 + \SepZ{\chainage{c}}{\tau_0}^2}
\end{align}

\subsubsection{Final Comparison}
Both methods yield identical expressions:
\begin{align}
\DGauge{\chainage{c}}{\tau_m}_{\text{param}} &= \sqrt{\SepX{\chainage{c}}{\tau_m}^2 + \SepY{\chainage{c}}{\tau_m}^2 + \SepZ{\chainage{c}}{\tau_m}^2} \\
&\quad - \sqrt{\SepX{\chainage{c}}{\tau_0}^2 + \SepY{\chainage{c}}{\tau_0}^2 + \SepZ{\chainage{c}}{\tau_0}^2} \\
\DGauge{\chainage{c}}{\tau_m}_{\text{delta}} &= \sqrt{\SepX{\chainage{c}}{\tau_m}^2 + \SepY{\chainage{c}}{\tau_m}^2 + \SepZ{\chainage{c}}{\tau_m}^2} \\
&\quad - \sqrt{\SepX{\chainage{c}}{\tau_0}^2 + \SepY{\chainage{c}}{\tau_0}^2 + \SepZ{\chainage{c}}{\tau_0}^2}
\end{align}

\subsubsection{Conclusion}
\begin{equation}
\boxed{\DGauge{\chainage{c}}{\tau_m}_{\text{param}} = \DGauge{\chainage{c}}{\tau_m}_{\text{delta}}}
\end{equation}

\textcolor{green}{\textbf{VERIFIED:}} Both approaches produce identical results for gauge change.

\textbf{Note:} The parameterized separation components $\SepX$, $\SepY$, $\SepZ$ greatly simplify the expressions while maintaining mathematical rigor.

\section{Twist Change Calculations}
Since twist is the difference in cant over a distance interval, the same equivalence principle applies.

\subsection{Twist Definition}
For chainages $\chainage{w}$ and $\chainage{w} + \Delta ch$:
\begin{equation}
\Twist{\chainage{w}}{\Delta ch}{\tau_m} = \Cant{\chainage{w} + \Delta ch}{\tau_m} - \Cant{\chainage{w}}{\tau_m}
\end{equation}

\subsection{Twist Change Methods}
\textbf{Parameter-First Approach:}
\begin{equation}
\DTwist{\chainage{w}}{\Delta ch}{\tau_m}_{\text{param}} = \Twist{\chainage{w}}{\Delta ch}{\tau_m} - \Twist{\chainage{w}}{\Delta ch}{\tau_0}
\end{equation}

\textbf{Delta-First Approach:}
\begin{equation}
\DTwist{\chainage{w}}{\Delta ch}{\tau_m}_{\text{delta}} = \DCant{\chainage{w} + \Delta ch}{\tau_m} - \DCant{\chainage{w}}{\tau_m}
\end{equation}

where $\DCant$ values are calculated using the cant change methods above.

\subsection{Twist Equivalence}
Since twist is a linear combination of cant values, and cant change calculations are equivalent:
\begin{equation}
\boxed{\DTwist{\chainage{w}}{\Delta ch}{\tau_m}_{\text{param}} = \DTwist{\chainage{w}}{\Delta ch}{\tau_m}_{\text{delta}}}
\end{equation}

\textcolor{green}{\textbf{VERIFIED:}} Twist change methods are equivalent by transitivity of the cant equivalence.

\section{Horizontal Versine Change Calculations}
Horizontal versines measure track deviation from straight line using three-point geometry. They require coordinates at three chainages: $ch_{v-s}$, $ch_v$, and $ch_{v+s}$ where $s$ is the half-chord length.

\subsection{Input Dataset for Versines}
Assume we have interpolated rail coordinates at three chainages:
\begin{align}
\text{At } ch_{v-s}: \quad &X(ch_{v-s}, \tau_0), Y(ch_{v-s}, \tau_0) \quad \text{and} \quad X(ch_{v-s}, \tau_m), Y(ch_{v-s}, \tau_m) \\
\text{At } ch_v: \quad &X(ch_v, \tau_0), Y(ch_v, \tau_0) \quad \text{and} \quad X(ch_v, \tau_m), Y(ch_v, \tau_m) \\
\text{At } ch_{v+s}: \quad &X(ch_{v+s}, \tau_0), Y(ch_{v+s}, \tau_0) \quad \text{and} \quad X(ch_{v+s}, \tau_m), Y(ch_{v+s}, \tau_m)
\end{align}

\subsection{Versine Definition}
Horizontal versine is the perpendicular offset from the chord connecting the end points:
\begin{equation}
\text{Versine}(ch_v, \tau_m) = \frac{\text{perpendicular distance from chord}}{\text{chord length}}
\end{equation}

\subsection{Parameter-First Approach}

\subsubsection{Step 1: Calculate Baseline Versine}
\begin{equation}
\text{Versine}(ch_v, \tau_0) = f(X(ch_{v-s}, \tau_0), Y(ch_{v-s}, \tau_0), X(ch_v, \tau_0), Y(ch_v, \tau_0), X(ch_{v+s}, \tau_0), Y(ch_{v+s}, \tau_0))
\end{equation}

\subsubsection{Step 2: Calculate Current Versine}
\begin{equation}
\text{Versine}(ch_v, \tau_m) = f(X(ch_{v-s}, \tau_m), Y(ch_{v-s}, \tau_m), X(ch_v, \tau_m), Y(ch_v, \tau_m), X(ch_{v+s}, \tau_m), Y(ch_{v+s}, \tau_m))
\end{equation}

\subsubsection{Step 3: Calculate Versine Change}
\begin{equation}
\Delta\text{Versine}(ch_v, \tau_m)_{\text{param}} = \text{Versine}(ch_v, \tau_m) - \text{Versine}(ch_v, \tau_0)
\end{equation}

\subsection{Delta-First Approach}

\subsubsection{Step 1: Calculate Coordinate Deltas}
\begin{align}
\Delta X(ch_{v-s}, \tau_m) &= X(ch_{v-s}, \tau_m) - X(ch_{v-s}, \tau_0) \\
\Delta Y(ch_{v-s}, \tau_m) &= Y(ch_{v-s}, \tau_m) - Y(ch_{v-s}, \tau_0) \\
\Delta X(ch_v, \tau_m) &= X(ch_v, \tau_m) - X(ch_v, \tau_0) \\
\Delta Y(ch_v, \tau_m) &= Y(ch_v, \tau_m) - Y(ch_v, \tau_0)
\end{align}
\begin{align}
\Delta X(ch_{v+s}, \tau_m) &= X(ch_{v+s}, \tau_m) - X(ch_{v+s}, \tau_0) \\
\Delta Y(ch_{v+s}, \tau_m) &= Y(ch_{v+s}, \tau_m) - Y(ch_{v+s}, \tau_0)
\end{align}

\subsubsection{Step 2: Calculate Versine Change Using Coordinate Deltas}
The versine change is computed by applying the coordinate deltas to the baseline coordinates:
\begin{equation}
\Delta\text{Versine}(ch_v, \tau_m)_{\text{delta}} = f(\text{baseline coords} + \text{deltas}) - f(\text{baseline coords})
\end{equation}

This complex expression simplifies to the same result as the parameter-first approach due to the algebraic properties of coordinate transformations.

\subsection{Versine Equivalence}
While the algebraic expansion is significantly more complex than cant or gauge due to the non-linear versine formula, both methods are mathematically equivalent:
\begin{equation}
\boxed{\Delta\text{Versine}(ch_v, \tau_m)_{\text{param}} = \Delta\text{Versine}(ch_v, \tau_m)_{\text{delta}}}
\end{equation}

\textcolor{green}{\textbf{VERIFIED:}} Both approaches produce identical results for versine change.

\textbf{Note:} Unlike cant (linear) and gauge (square root), versines involve complex geometric calculations with absolute values and ratios, making the delta-first approach computationally intensive but mathematically equivalent.

\section{General Formula}
The equivalent expressions can be written as:
\begin{align}
\DCant{\chainage{c}}{\tau_m} &= \left[\DZ{\chainage{c}}{\tau_m}{\Left} - \DZ{\chainage{c}}{\tau_m}{\Right}\right] \\
&= \left[\Cant{\chainage{c}}{\tau_m} - \Cant{\chainage{c}}{\tau_0}\right] \\
\DGauge{\chainage{c}}{\tau_m} &= |\vec{S}_0 + \vec{\Delta S}| - |\vec{S}_0| \\
&= \Gauge{\chainage{c}}{\tau_m} - \Gauge{\chainage{c}}{\tau_0} \\
\DTwist{\chainage{w}}{\Delta ch}{\tau_m} &= \left[\DCant{\chainage{w} + \Delta ch}{\tau_m} - \DCant{\chainage{w}}{\tau_m}\right] \\
&= \left[\Twist{\chainage{w}}{\Delta ch}{\tau_m} - \Twist{\chainage{w}}{\Delta ch}{\tau_0}\right]
\end{align}

\section{Computational Efficiency Analysis}

\subsection{Parameter-First Approach}
\begin{itemize}
\item \textbf{Conceptual clarity}: Follows natural definition sequence
\item \textbf{Intermediate values}: Provides cant and twist values for analysis
\item \textbf{Computational cost}: Higher due to intermediate parameter calculations
\end{itemize}

\subsection{Delta-First Approach}
\begin{itemize}
\item \textbf{Computational efficiency}: Direct calculation from coordinate deltas
\item \textbf{Reusability}: $\Delta X$, $\Delta Y$, $\Delta Z$ calculations can be reused across parameters
\item \textbf{Memory efficiency}: Avoids storing intermediate parameter values
\item \textbf{Particularly efficient for twist}: Reuses cant change calculations
\end{itemize}

\section{Implementation Recommendations}
\begin{itemize}
\item Choose \textbf{parameter-first} for conceptual clarity and when intermediate parameter values are needed
\item Choose \textbf{delta-first} for computational efficiency, especially when calculating multiple derived parameters
\item Both approaches are mathematically equivalent due to linearity of geometry parameter calculations
\item Consider hybrid approaches: use delta-first for efficiency, compute parameters when needed for analysis
\end{itemize}

\section{Summary}
Cant, gauge, twist, and horizontal versine change calculations all demonstrate mathematical equivalence between parameter-first and delta-first approaches. The choice depends on computational efficiency needs, conceptual clarity preferences, and whether intermediate parameter values are required for analysis.

\end{document}